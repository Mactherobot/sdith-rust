\documentclass[twoside,11pt]{report}

\usepackage[latin1]{inputenc}
\usepackage[american]{babel}
\usepackage{a4}
\usepackage{latexsym}
\usepackage{amssymb}
\usepackage{algorithm}
\usepackage{algpseudocode}
\usepackage[outputdir=out]{minted}
\setminted[rust]{breaklines}
\usepackage[]{amsmath}
\usepackage{epsfig}
\usepackage[T1]{fontenc}
\usepackage{color}
\usepackage{epstopdf}
\usepackage{microtype}
\usepackage{hyperref}
\usepackage[useregional]{datetime2}
\DTMlangsetup[en-US]{showdayofmonth=false}
\usepackage{lipsum}
\usepackage{ctable} % for \specialrule command
\usepackage{enumitem}
\usepackage{stmaryrd}
\usepackage{wasysym}

\usepackage{listings}
\lstdefinestyle{tree}{
  literate=
  {├}{{\smash{\raisebox{-1ex}{\rule{1pt}{\baselineskip}}}\raisebox{0.5ex}{\rule{1ex}{1pt}}}}1 
  {─}{{\raisebox{0.5ex}{\rule{1.5ex}{1pt}}}}1 
  { }{~}1
  {└}{{\smash{\raisebox{0.5ex}{\rule{1pt}{\dimexpr\baselineskip-1.5ex}}}\raisebox{0.5ex}{\rule{1ex}{1pt}}}}1 
  {│}{{\smash{\raisebox{-1ex}{\rule{1pt}{\baselineskip}}}\raisebox{0.5ex}{\rule{1ex}{0pt}}}}1 
}
\setlength{\parindent}{10pt} % indentation

\usepackage{amsthm}
\theoremstyle{definition}
\newtheorem{definition}{Definition}[section]
\newcommand{\definitionautorefname}{Definition}

\theoremstyle{plain}
\newtheorem{lemma}{Lemma}[section]

\addto\extrasamerican{%
  \def\chapterautorefname{Chapter}%
  \def\sectionautorefname{Section}%
  \def\subsectionautorefname{Section}%
  \def\lemmaautorefname{Lemma}%
  \def\figureautorefname{Figure}%
  \def\algorithmautorefname{Algorithm}%
}

\makeatletter
\newenvironment{breakablealgorithm}
  {% \begin{breakablealgorithm}
   \begin{center}
     \refstepcounter{algorithm}% New algorithm
     \hrule height.8pt depth0pt \kern2pt% \@fs@pre for \@fs@ruled
     \renewcommand{\caption}[2][\relax]{% Make a new \caption
       {\raggedright\textbf{\fname@algorithm~\thealgorithm} ##2\par}%
       \ifx\relax##1\relax % #1 is \relax
         \addcontentsline{loa}{algorithm}{\protect\numberline{\thealgorithm}##2}%
       \else % #1 is not \relax
         \addcontentsline{loa}{algorithm}{\protect\numberline{\thealgorithm}##1}%
       \fi
       \kern2pt\hrule\kern2pt
     }
  }{% \end{breakablealgorithm}
     \kern2pt\hrule\relax% \@fs@post for \@fs@ruled
   \end{center}
  }
\makeatother



\newtheoremstyle{lemma}% <name>
{3pt}% <Space above>
{3pt}% <Space below>
{\normalfont}% <Body font>
{}% <Indent amount>
{\upshape}% <Theorem head font>
{:}% <Punctuation after theorem head>
{.5em}% <Space after theorem headi>
{}% <Theorem head spec (can be left empty, meaning `normal')>

\renewcommand*\sfdefault{lmss}
\renewcommand*\ttdefault{txtt}

\newcommand{\todo}[1]{{\color[rgb]{.5,0,0}\textbf{$\blacktriangleright$#1$\blacktriangleleft$}}}

\renewcommand{\algorithmicrequire}{\textbf{Input:}}
\renewcommand{\algorithmicensure}{\textbf{Output:}}


\begin{document}
\newcommand{\rust}[1]{{\mintinline{rust}|#1|}}\

%%%%%%%%%%%%%%%%%%%%%%%%%%%%%%%%%%%%%%%%%%%%%%%%%%%%%%%%%%%%%%%%%%%%%%%

\pagestyle{empty}
\pagenumbering{roman}
\vspace*{\fill}\noindent{\rule{\linewidth}{1mm}\\[4ex]
{\Huge\sf SD-in-the-Head rust implementation and optimization}\\[2ex]
{\huge\sf Hugh Benjamin Zachariae, 201508592 \\ Magnus Jensen,
201708626}\\[2ex]
\noindent\rule{\linewidth}{1mm}\\[4ex]
\noindent{\Large\sf Master's Thesis, Computer Science\\[1ex]
  \today \\[1ex] Advisor: Diego F. Aranha\\[15ex]}\\[\fill]}
\epsfig{file=images/logo.eps}\clearpage

% A math share macro
\newcommand{\sh}[1] {\ensuremath{\llbracket #1 \rrbracket}}

%%%%%%%%%%%%%%%%%%%%%%%%%%%%%%%%%%%%%%%%%%%%%%%%%%%%%%%%%%%%%%%%%%%%%%%

\pagestyle{plain}
\chapter*{Abstract}
\addcontentsline{toc}{chapter}{Abstract}

\todo{in English\dots}

\chapter*{Resum\'e}
\addcontentsline{toc}{chapter}{Resum\'e}

\todo{in Danish\dots}

\chapter*{Acknowledgments}
\addcontentsline{toc}{chapter}{Acknowledgments}

\todo{\dots}

\vspace{2ex}
\begin{flushright}
  \emph{Hugh Benjamin Zachariae and Magnus Jensen}\\
  \emph{Aarhus, \today.}
\end{flushright}

\tableofcontents
\cleardoublepage
\pagenumbering{arabic}
\setcounter{secnumdepth}{2}

%%%%%%%%%%%%%%%%%%%%%%%%%%%%%%%%%%%%%%%%%%%%%%%%%%%%%%%%%%%%%%%%%%%%%%%

\chapter{Introduction}\label{ch:intro}

few pages. Introduce what we have done and how the paper is structured

\section{Post-Quantum Cryptography}\label{sec:quantum}

Digital signatures are essential for ensuring a secure internet. As our collective reliance on the internet grows, they play a critical role in identifying and verifying the authenticity of the sources we interact with over the network. Digital signatures ensure that software or web pages come from trusted sources and have not been tampered with, forming a cornerstone of IT security. They also enable the verification of the signer's identity and help detect unauthorized changes to data. Additionally, digital signatures support non-repudiation, meaning they provide proof to a third party that a signature was genuinely created by the signer.

The most widely used digital signature schemes rely on public-key cryptography. Each signer has a public and a private key. The private key allows the signer to sign messages, while the public key is used to verify the signature. The security of a signature is grounded in the computational complexity of solving specific mathematical problems. Standardization efforts in the early 2000s~\cite{pub2000digital} led to the adoption of algorithms such as RSA, DSA, and ECDSA, which leverage intractability of factoring large integers and solving discrete logarithms to ensure robust protection.

\subsection{NIST call to action}

In 1994, the discovery of Shor's algorithm~\cite{shor1997} introduced a polynomial-time method for solving these problems using quantum computing. At that time, quantum computers were theoretical constructs. However, in 2001, Chuang et al.~\cite{vandersypen2001experimental,buchmann2004post} successfully implemented Shor's algorithm on a 7-qubit quantum computer. Two decades later, significant advancements in quantum technology, have brought us closer to the point where quantum computers could be a reality.
\todo{Add some examples of the significant advancements in quantum technology.}

A necessary pre-requisite for a post-quantum secure signature scheme is the existence of a problem that is intractable for quantum computers. Several candidates from the family of NP-hard problems within complexity theory have been proposed. These are problems that can not be solved in polynomial-time by any classical or quantum computer.

In 2016, the National Institute of Standards and Technology (NIST) initiated a call for quantum-resistant public-key cryptosystems~\cite{nistcall}. This effort led to the standardization of three signature schemes: Dilithium~\cite{ducas2018crystals}, Falcon~\cite{fouque2018falcon}, and SPHINCS+~\cite{bernstein2019sphincs+}. Dilithium and Falcon are lattice-based schemes that rely on the hardness of structured lattice problems. They offer efficient key generation, signing, and verification, making them practical for many applications. However, their shared dependence on the same mathematical assumption poses a risk if this assumption is broken.

In contrast, SPHINCS+ is a hash-based signature scheme, relying solely on the security of cryptographic hash functions rather than lattice-based problems. This makes it an important alternative, providing diversity in case lattice assumptions are compromised. Despite its robustness, SPHINCS+ is significantly slower than Dilithium and Falcon, with high computational costs and larger signature sizes, which limits its suitability for real-world applications.

To mitigate the risks of relying heavily on lattice-based schemes, NIST issued a second call in 2022 for quantum-resistant signature schemes based on alternative assumptions, ensuring a more diversified and resilient set of standards for the future.


\subsection{Code-based problems}

With the introduction of public-key cryptosystems by Diffie and Hellman in 1976, we saw the introduction to schemes like RSA, that is based on factoring of large integers and ElGamal which is based on the discrete logarithm problem. While less known, there exists schemes based on coding theory -- its own discipline within information theory which predates the introduction of public-key cryptography. In 1978, McEliece introduced a PK scheme based on coding theory~\cite{mceliece1978public} -- only two years later -- based on the hardness of decoding random linear codes~\cite{berlekamp1978inherent}.

While the scheme never saw widespread adoption due to its large key sizes, the interest in code-based crypto-schemes has seen a resurgence as it is not vulnerable to known quantum attacks like Shor's algorithm. Recent developments in code-based cryptosystems have shown that the performance needed for code based cryptographic schemes practically is within reach. Introductions of linear based codes on finite fields and MPCitH (Multi-Party Computation in the Head) threshold schemes~\cite{baum2020concretely} show promising results. With the SD-in-the-Head scheme~\cite{aguilarsyndrome11,feneuil2023threshold} presented in this report, such techniques are combined to define an efficient post-quantum secure signature scheme.

\section{Our contributions}

\todo{describe how we progressed through the project in an interesting way. What did we set out to do, what were the hurdles.}

In this report, we present an implementation of the SD-in-the-Head protocol in Rust. While the NIST standardization process initially prioritizes selecting the appropriate protocols, the ultimate goal is to produce a well-documented specification that enables implementors to safely and accurately reproduce the chosen protocols.

To this end, it is crucial for both NIST and the authors to provide an in-depth specification accompanied by clear, well-documented examples to facilitate a smooth transition to the new standard.

Our objectives include delivering a high-performance, thoroughly documented implementation, complemented by extensive test coverage to ensure correctness and security. Additionally, we aim to identify and address potential pitfalls and challenges encountered during the implementation process in order to provide this feedback to the authors. Furthermore, we will attempt to condense the literature behind the \emph{SD-in-the-Head} protocol and present a comprehensive report on the protocol and its prerequisites.

Finally, this project serves as an opportunity to test our capabilities by implementing a cutting-edge cryptographic protocol in a modern, memory-safe language like Rust. As the final product, we will deliver this report alongside a Rust library and a client implementation for the SD-in-the-Head protocol.


\subsection{Choosing the SD-in-the-Head protocol}
The SD-in-the-Head protocol offers a highly modular design, enabling extensive configurability and support for various subroutines. This flexibility is particularly evident in the two variants included in the specification~\cite{aguilarsyndrome11}: the \textit{hypercube} and \textit{threshold} variants, both of which demonstrate significant performance improvements over the original protocol~\cite{feneuil2022syndrome,aguilar2023return,feneuil2023threshold}.

The threshold variant, in particular, provides an opportunity to explore and implement several cryptographic primitives, including Merkle Trees~\cite{becker2008merkle}, Linear Secret Sharing (Shamir), and Multi-Party Computation in the Head (MPCitH)~\cite{baum2020concretely}. Along with Galois Fields~\cite{brownadvanced}.

Additionally, the authors supply a detailed and well-crafted specification, along with a reference implementation in C++, which serves as a robust starting point for our work.


\subsection{Choosing Rust}
We chose to implement the SD-in-the-Head protocol in Rust for several compelling reasons. Rust, as a modern systems programming language, offers inherent memory and thread safety without compromising performance. Its strong safety guarantees are a key reason why Rust is included in the NIST list of safer programming languages~\cite{nistsaferlanguages}. These properties are particularly critical when implementing cryptographic primitives, where memory safety and robustness against concurrency issues are foundational to secure software development. Additionally, Rust provides fine-grained control over hardware resources, enabling the optimization of performance -- an essential consideration in cryptographic implementations.

Our choice of Rust was also driven by a desire to investigate its strengths and limitations in the context of implementing and optimizing cryptographic protocols. To this end, we prioritized code readability and maintainability. For maintainability, we incorporated rigorous testing frameworks and benchmarking tools to validate the correctness and efficiency of the implementation. These measures also facilitate meaningful comparisons with alternative signature schemes. For readability, we emphasized clear, well-documented, and logically structured code.

Given the inherent flexibility and modular nature of the SD-in-the-Head protocol, our implementation leverages Rust's feature flags to enhance modularity and interchangeability. This approach not only supports a high degree of configurability but also ensures that the implementation remains adaptable to future changes and extensions.

\todo{Other signature schemes. How many signatures can be generated per second.}

\section{Structure of the report}
The report is structured as follows: \autoref{ch:prelim} provides the necessary background information on the cryptographic primitives and sub-protocols used in the SD-in-the-Head protocol. \autoref{ch:spec} presents the specification of the SD-in-the-Head protocol, focusing on the threshold variant. \autoref{ch:impl} details our implementation of the protocol in Rust, including the development process, design choices, and optimizations. \autoref{ch:bench} evaluates the performance of our implementation and compares it to the reference implementation and other alternatives. Finally, \autoref{ch:conclusion} summarizes our contributions.

\todo{Keep this up to date}

%%%%%%%%%%%%%%%%%%%%%%%%%%%%%%%%%%%%%%%%%%%%%%%%%%%%%%%%%%%%%%%%%%%%%%%

\chapter{Preliminaries}\label{ch:prelim}

In this section, we outline the mathematical and cryptographic foundations necessary to understand the SD-in-the-Head scheme and its components, providing a step-by-step progression from the basics to the construction of the protocol.

We begin by introducing the essential subroutines that underpin the SD-in-the-Head protocol. This includes \textbf{Galois Finite Fields}~\cite{martinez2023syndromes, reed1960polynomial, brownadvanced} along with cryptographic primitives; \textbf{collision-resistant hash functions} and \textbf{Merkle tree commitment scheme}~\cite{becker2008merkle}. In addition, we provide short introductions to the foundational protocols of \textbf{Secure Multi-Party Computation} (MPC) and \textbf{Zero-Knowledge} (ZK) proofs, which serve as the underlying cryptographic protocols.

Next, we delve into the Syndrome Decoding (SD) problem, which forms the backbone of the SD-in-the-Head protocol. This includes an overview of the problem definition~\cite{aguilarsyndrome11, mceliece1978public, berlekamp1978inherent, baldi2013optimization}, and its polynomial representation. This will support an understanding of how the protocol operates on hard instances of the SD problem.

Building upon these foundations, we lay the groundwork for the construction of the SD-in-the-Head protocol. First, we discuss the \textbf{Multi-Party-Computation-in-the-Head} (MPCitH) paradigm, explaining how it is leveraged to construct efficient ZK proofs~\cite{ishai2007zero}. Additionally we detail the role of the MPC preprocessing~\cite{baum2020concretely} and linear secret-sharing scheme~\cite{feneuil2023threshold}, which dramatically improve the efficiency of the MPCitH construction, and verification. Next, we introduce the \textbf{Fiat-Shamir heuristic}, a powerful tool that transforms an interactive ZK protocol into a non-interactive signature scheme~\cite{fiat1986prove}.

\section{Galois Finite Field}\label{sec:gf256}
Finite fields lays the basis for many cryptographic protocols and primitives. In essence, finite field theory describes the relation and properties between numbers, while disregarding the numbers themselves. In order to define finite fields, we first need to define a \textit{group} and a \textit{finite field} (\autoref{def:group} and \autoref{def:field}, respectively).

\begin{definition}\label{def:group}
  A group $\mathbb{G}$ is a tuple $(\mathbb{G}, \times, 1)$ where, $\mathbb{G}$ is the set of elements in the group, $\times$ is a binary operator, and $1$ is the multiplicative identity. The group $\mathbb{G}$ must satisfy the following properties:
  \begin{enumerate}
    \item $\mathbb{G}$ is \textbf{closed} under $\times$. For all $a,b \in \mathbb{G}$, $a \times b \in \mathbb{G}$.
    \item $\mathbb{G}$ is \textbf{associative} under $\times$. For all $a,b,c \in \mathbb{G}$, $(a \times b) \times c = a \times (b \times c)$.
    \item $\mathbb{G}$ has an \textbf{identity} element $1$ under $\times$. For all $a \in \mathbb{G}$, $a \times 1 = a$.
    \item $\mathbb{G}$ has an \textbf{inverse} element for each element under $\times$. For all $a \in \mathbb{G}$, there exists an element $a^{-1}$ such that $a \times a^{-1} = 1$.
  \end{enumerate}
\end{definition}

\begin{definition}\label{def:field}
  A finite field $\mathbb{F}$ is a tuple $(\mathbb{F}, +, \times, 0, 1)$ where, $\mathbb{F}$ is the set of elements in the field, $+$ and $\times$ are the addition and multiplication operators, and $0$ and $1$ are the additive and multiplicative identities. The field $\mathbb{F}$ must satisfy the following properties:
  \begin{enumerate}
    \item  $+$ and $\times$ are \textbf{commutative}, \textbf{associative} and $\times$ is \textbf{distributive} over $+$.
          \begin{enumerate}
            \item $\forall a,b \in \mathbb{F}: a + b = b + a$ and $a \times b = b \times a$ (commutative)
            \item $\forall a,b,c \in \mathbb{F}: (a + b) + c = a + (b + c)$ and $(a \times b) \times c = a \times (b \times c)$ (associative)
            \item $\forall a,b,c \in \mathbb{F}: a \times (b + c) = a \times b + a \times c$ (distributive)
          \end{enumerate}
    \item $(\mathbb{F}, +, 0)$ forms an additive group
    \item $(\mathbb{F} \setminus \{0\}, \times, 1)$ forms a multiplicative group
  \end{enumerate}
\end{definition}

\noindent
The most commonly known finite fields are the prime fields $\mathbb{F}_p$, widely used in discrete logarithm cryptosystems, such as ECDSA and ElGamal. These fields can be extended into $\mathbb{F}_{p^n}$, containing $p^n$ elements which we will describe in \autoref{sub:field_extension}. However, prime fields are not efficient for iterating over large numbers, due to the modular reduction overhead, a common requirement in protocols such as SD-in-the-Head. For better efficiency, we require a field that can be represented using bits or bytes. One method to achieve this is by constructing fields commonly known as Galois fields. Consider the simplest prime field, also called binary field, $\mathbb{F}_2 = ({0,1}, \texttt{XOR}, \texttt{AND}, 0, 1)$. It is straightforward to verify that this field satisfies the properties outlined in \autoref{def:field}.

The Galois field $\mathbb{F}{2^n}$ is an extension field of $\mathbb{F}2$, defined as $\mathbb{F}{2^n}$. This field is constructed by selecting an irreducible polynomial $f(x)$ of degree $n$ with coefficients in $\mathbb{F}2$. The elements of $\mathbb{F}{2^n}$ are represented as polynomials of degree at most $n-1$ with coefficients in $\mathbb{F}2$. Consequently, $\mathbb{F}{2^n}$ contains exactly $2^n$ elements. The polynomial $f(x)$ ensures that $\mathbb{F}_{2^n}$ forms a valid field as it defines the modular reduction in multiplication.

Each element in $\mathbb{F}_{2^n}$ is represented as a polynomial of degree at most $n-1$ with coefficients in $\mathbb{F}_2$.

\begin{align*}
  a(x) = a_{n-1}x^{n-1} + a_{n-2}x^{n-2} + \cdots + a_1x + a_0
\end{align*}

\noindent
where $a_i \in \mathbb{F}_2$ This corresponds directly into the 8-bit binary representation $a_{n-1}a_{n-2}\dots a_1a_0$. The addition and multiplication operations within $\mathbb{F}{2^n}$ are defined as follows:

\begin{itemize}
  \item \textbf{Addition}: The addition of two elements $a,b \in \mathbb{F}_{2^n}$ is defined as the bitwise XOR of the two elements. Due to the fact that it corresponds to addition modulo $2$.
  \item \textbf{Multiplication}: The multiplication of two elements $a,b \in \mathbb{F}_{2^n}$ is defined as the multiplication of the two polynomials modulo the irreducible polynomial $p(x)$.
\end{itemize}

A common choice for the irreducible polynomial is the Rijndael polynomial $p(x) = x^8 + x^4 + x^3 + x + 1$ used in the AES protocol~\cite{brownadvanced}. This field is commonly known as $\mathbb{F}_{2^8}$. The size of the field allows us to represent elements as bytes, which is efficient for iterating over large numbers of digits and it contains all the elements of a byte.

\subsubsection{Field extension}\label{sub:field_extension}

An advantageous property of Galois fields is that they can easily be extended, like $\mathbb{F}_{2}$ to $\mathbb{F}_{2^8}$. To ensure security in the SD-in-the-Head protocol, the field $\mathbb{F}_{2^8}$, which we shall denote as $\mathbb{F}_q$ is extended to encompass the field 32-bit unsigned integers. This can be done as a \textit{tower extension} by first building a degree-2 extension to $\mathbb{F}_{q^2}$ using the irreducible polynomial $F_{q^2} = F_q[X] / (X^2 + X + 32)$ and finally extending this field to $\mathbb{F}_{q^4}$ using the irreducible polynomial $F_{q^4} = F_q[Z] / (Z^2 + Z + 32(X))$. While you could create the final field independently, the tower extension allows us to reuse the construction of the intermediate field. Addition and multiplication in the extension fields are represented as the following

\begin{align*}
  a         & = a_0 + a_1x                                               &  & a \in \mathbb{F}_{q^\eta}, a_0, a_1 \in \mathbb{F}_q \\
  b         & = b_0 + b_1x                                               &  & b \in \mathbb{F}_{q^\eta}, b_0, b_1 \in \mathbb{F}_q \\\\
  a + b     & = (a_0 + a_1x) + (b_0 + b_1x) = (a_0 + b_0) + (a_1 + b_1)x                                                           \\
  a \cdot b & = (a_0 + a_1x) \cdot (b_0 + b_1x)                                                                                    \\
            & = a_0b_0 + (a_0b_1 + a_1b_0)x + a_1b_1x^2                                                                            \\
            & = a_0b_0 + (a_0b_1 + a_1b_0)x + a_1b_1(x + 32)                                                                       \\
            & = a_0b_0 + 32a_1b_1 + (a_0b_1 + a_1b_0 + a_1b_1)x
\end{align*}

\section{Cryptographic Primitives and protocols}
We assume the reader has basic knowledge of cryptographic primitives but will provide brief introductions to the primitives used in the SD-in-the-Head protocol. If you know the basics feel free to skip this section.

\subsection{Collision Resistant Hash Functions}\label{sec:prelim_hash}
We denote a hash function by a generator $\mathcal{H}$ which on input of a security parameter $k$ outputs a function $h : \{0,1{\}}^* \rightarrow \{0,1{\}}^{2k}$. We denote a hash function $h$ to be cryptographically secure if it satisfies the following properties

\begin{lemma}[Preimage Resistance]\label{lem:preimage}
  Given a hash function $h$ and a hashed message $c$, there exist no PPT algorithm that can find a message $m$ such that $h(m) = c$ with non-negligible probability.
\end{lemma}

\begin{lemma}[Collision Resistance]\label{lem:collision}
  Given a hash function $h$, there exist no PPT algorithm that can find two distinct messages $m, m'$ such that $h(m) = h(m')$ with non-negligible probability.
\end{lemma}

The consensus among researchers indicates that for collision and preimage resistance, quantum computing is expected to reduce the security level from $k$ to $k/2$ through approaches like Grover's algorithm~\cite{nielsen2010quantumgrover}. In contrast, no known methods have been proposed to reduce the security level for collision resistance. Consequently, protocols that depend on collision-resistant hash functions, such as Merkle Signature or SD-in-the-Head, remain a viable choice for post-quantum secure protocols.

In proving the security of hash functions, researchers often resort to models like the Quantum Random Oracle Model (QROM), while there is significant work being done to further prove the future security of hash functions~\cite{dtuPostquantumSecurity}.

\subsubsection{Extendable-output (hash) functions (XOF)}\label{sec:xof}
\todo{Give a short introduction to XOF}

\subsection{Commitment Schemes}
A commitment scheme allows a \textit{prover} to \textit{commit} to a value and later \textit{open} or reveal the value to a \textit{verifier}. It satisfies the following two properties:

\begin{lemma}[Binding]\label{lem:binding}
  Given a commitment $c$ to a value $v$, a dishonest prover can only change the committed value and have the verifier accept the new value with negligible probability.
\end{lemma}
\begin{lemma}[Hiding]\label{lem:hiding}
  Given a commitment $c$ to a value $v$, the verifier can only recover the value $v$ with negligible probability.
\end{lemma}

Consider a simple scenario. Alice and Bob are playing a game of \textit{heads or tails}. However each does not trust the other not to cheat
\begin{enumerate}[parsep=0pt, itemsep=0pt]
  \item Bob does not trust Alice not to correct her guess after learning his result
  \item Alice does not trust Bob not to use the knowledge of her guess to cheat
\end{enumerate}
Therefore, consider the following protocol. Given a public secure hash function $h$ \autoref{sec:prelim_hash}:
\begin{enumerate}[parsep=0pt, itemsep=0pt]
  \item Alice chooses a guess $g \in \{0,1\}$ for \textit{tails} and \textit{heads} respectively.
  \item Alice computes her commitment $c = h(g | w)$ for some random string $w$ and sends $c$ to Bob.
  \item Bob flips his coin to get the result $r \in \{0,1\}$ for \textit{tails} and \textit{heads} respectively.
  \item Bob announces $r$ to Alice.
  \item Alice announces $g', w$ to Bob.
  \item Bob accepts the result if $h(g' | w) = c$.
\end{enumerate}

From \autoref{lem:preimage}, we can see that Bob is not able to learn $g$ before he flips his coin without knowing $g \mid w$.\footnote{We can adjust the length of $w$ to enhance the security of the protocol.} This ensures \autoref{lem:hiding}.

Furthermore, Alice cannot change her result $g$ without being able to find a different $w'$ such that:
\[
  h(g' \mid w') = c.
\]
which she can only do with negligible probability per \autoref{lem:collision}, ensuring \autoref{lem:binding}.

\subsubsection{Merkle Tree Commitment Scheme}
A variant of such a scheme which allows for partial opening, meaning that the prover does not have to open the entire commitment, but can instead open a subset of the commitment, is the \textit{Merkle Tree Commitment scheme}~\cite{becker2008merkle}.

Let $v_1, \dots, v_n$ be the values to be committed, and let $h$ denote a cryptographically secure hash function. The leaves of the Merkle tree, $a_{0,1}, \dots, a_{0,n}$, are initialized as the hash values of the input values:
\[
  a_{0,i} = h(v_i), \quad \text{for } i = 1, \dots, n.
\]

Each parent node $a_{i,j}$, where $i$ represents the level of the tree and $j$ denotes the parent index at that level, is computed as the hash of its two child nodes:
\[
  a_{i,j} = h(a_{i-1,2j} \, | \, a_{i-1,2j+1}),
\]
where $|$ indicates concatenation of the child nodes.

The root node, $a_{h,0}$, where $h$ is the height of the tree, serves as the commitment value and is sent to the verifier.

To reveal a committed value $v$, the prover computes the authentication path for the value's leaf node $a_{0,l}$, where $l$ is the index of the value. The authentication path $A$ consists of the sequence of sibling nodes required to reconstruct the root. The first element in the authentication path, $A_0$, is the sibling of the leaf node:
\[
  A_0 =
  \begin{cases}
    a_{0,l-1}, & \text{if } l \text{ is odd,}  \\
    a_{0,l+1}, & \text{if } l \text{ is even.}
  \end{cases}
\]
This element is recorded as $auth_0$.

For higher levels, each subsequent element in the path is determined as:
\[
  A_i = h(A_{i-1} \, | \, auth_{i-1}),
\]
where $auth_i$ is the sibling node of $A_{i-1}$ at the same level.

The prover transmits the authentication path $A$ and the index $p$ of the selected value to the verifier. The verifier checks the validity of the commitment by using the provided leaf value and authentication path to recompute the root of the Merkle tree.

\subsubsection{Security properties}

The hiding property of the protocol is ensured by the use of a secure hash function, as the prover only shares the root of the tree. If the adversary were able to recover any value $v$, this would break the preimage resistance \autoref{lem:preimage} of the hash function.

Conversely, the binding property \autoref{lem:binding} is upheld by the fact that the adversary would have to find a collision for the hash function to change the commitment value.

A notable property of the Merkle tree is its ability to open a partial subset of the committed values by providing only the necessary authentication paths for the values in the subset. The verifier can still validate these values using a single global root value.

This property is particularly advantageous in protocols like the SD-in-the-Head threshold protocol, where only a subset of the committed values needs to be opened while keeping the rest concealed. By leveraging this feature, the prover can efficiently prove the validity of the subset without revealing the entire set of commitments.

\subsection{Secret Sharing Schemes (SSS)}

Secret Sharing Schemes (SSS) are a class of cryptographic primitives that allow a secret to be dispersed into multiple \textit{shares}, each of which can be distributed to different parties. These shares can then be recombined to reconstruct the original secret. SSS was initially invented~\cite{shamir1979share} to alleviate the single-point-of-failure problem for storage. Below we supply a definition of a secret sharing scheme~\cite{cramer2015secure} and the notation used in this report.

\begin{definition}[Secret Sharing Scheme (SSS)]\label{def:ss-share}
  We denote a SSS be a system of ``dispersing'' a secret $s$ into a sequence of $N$ pieces of data $\sh{s} = [s_1, \dots, s_n]$, called \emph{shares}, such that we have the following properties:
  \begin{itemize}[parsep=0pt, itemsep=0pt]
    \item \textit{$(N-1)$-privacy}: An adversary holding $l < N$ shares learns nothing about the secret $s$.
    \item \textit{Reconstruction}: Given a subset of shares $s_i$ of size $\ell$, the scheme can reconstruct the secret $s = \texttt{open}(s_i)$.
  \end{itemize}
  We denote the following notation:
  \begin{itemize}
    \item \texttt{share}: We denote a the function \texttt{share}$_n(s) \rightarrow \sh{s}$ to be a function that takes a secret $s$ and returns a sharing $\sh{s}$ consisting of $n$ shares $s_i$ such that \texttt{open}$(\sh{s}) = s$.
    \item \texttt{open}: We denote a function \texttt{open}$(\sh{s}) \rightarrow s$ to be the opening of the shared value $\sh{s}$ where each party $P_i$ broadcasts its share $s_i$ allowing all participants to recover the secret value $s$.
  \end{itemize}
\end{definition}

\subsubsection{Additive SSS and additive homomorphism}\label{sec:additive-sss}

In its simple form, a SSS can be based on the same ideas as the \textit{one-time pad} encryption. Assume a Abelian group $F$ and a share $\sh{s} \in F^n$ of a secret $s \in F$. Sample $n-1$ uniformly random elements $[s_1, \dots, s_{n-1}] \in \mathbb{F}^{n-1}$ and set $s_n = s - \sum_{i=1}^{n-1} s_i$. Then we have that
\begin{align*}
  s = \sum_{i=1}^{n} s_i
\end{align*}
A notable feature of this construction is that if the parties possess two shares, $\sh{x}$ and $\sh{y}$, they can easily compute $\sh{x + y}$ by simply adding their individual shares. This property is referred to as \textit{partial homomorphic} or \textit{additively homomorphic}. The correctness of $\sh{x} + \sh{y} = \sh{c}$ for $c = x + y$ can be demonstrated as follows:
\begin{align*}
  c & = \texttt{open}(\sh{c})                         \\
    & = \texttt{open}(\sh{x} + \sh{y})                \\
    & = \texttt{open}([x_i + y_i])                    \\
    & = \sum_{i=1}^{n} x_i + y_i                      \\
    & = \sum_{i=1}^{n} x_i + \sum_{i=1}^{n} y_i       \\
    & = \texttt{open}(\sh{x}) + \texttt{open}(\sh{y}) \\
    & = x + y
\end{align*}
This property enables us to easily produce Multi-Party Computation (MPC) (see \autoref{sec:mpc}) on top of such a SSS\@. However, to allow full arithmetic circuits, we need more elaborate protocols, which we will discuss in \autoref{sec:zk-arithmetic-mpc-circuits}.
\subsubsection{Threshold SSS}\label{sec:threshold-sss}

In many situations such schemes are not sufficient. Imagine wanting to solve the the distribution of a secret (e.g. a password) among a set of locations to ensure that you do not have a single-point-of-failure. To solve this, we require a SSS that can reconstruct the secret from a a subset of shares up to a threshold $\ell$. This is called a $(\ell, n)$-threshold SSS\@.

\begin{definition}[$(\ell, n)$-private (threshold) SSS]\label{def:mpc-ss-threshold}
  A $(\ell, n)$-threshold SSS~\cite{cramer2015secure}, where $\ell, n$ are integers with $0 \leq \ell < n$, denotes a SSS such that:
  \begin{itemize}[parsep=0pt, itemsep=0pt]
    \item \textit{$\ell$-privacy}: Any $\leq t$ shares jointly give no information on the secret $s$. Namely,
    \item \textit{Reconstruction}: Any $\geq t + 1$ of these shares jointly determine the secret $s$ uniquely (\textit{$(\ell + 1)$-opening}).
  \end{itemize}
\end{definition}

In this case, the SSS described above would not suffice as loosing even one share would result in the loss of the secret. There are several ways to solve this problem. We could distribute more than one share to each location. If distributed correctly, we can ensure that only $t \leq N$ parties need to be available. However, there are also other more sophisticated schemes that can be used to ensure that the secret is not lost.

\subsubsection{Shamir's Secret Sharing}\label{sec:shamir}

One particular example, and the one utilized by the SD-in-the-Head protocol, is Shamir's Secret Sharing scheme~\cite{shamir1979share,cramer2015secure}. This $(\ell, n)$-private threshold SSS is based on polynomials over a finite field $\mathbb{F}$. A secret $s$ is shared by choosing a random polynomial $P_s$ of degree at most $\ell \geq 2$ with $s = P_s(0)$. For each participant $i \in [n]$, $P_i$, a share $s_i$ is generated by evaluating $P(i)$. The secret can then be reconstructed by interpolating the polynomial from a $\ell + 1$ shares and with $\ell$ or less shares we gain no information about the secret -- i.e. we have $\ell$-privacy. We can prove reconstruction, correctness and privacy using \textit{lagrange interpolation}.

\begin{definition}[Lagrange interpolation]\label{def:lagrange}
  If $h(X)$ is a polynomial over $\mathbb{F}$ of degree at most $\ell$ and if $C$ is a subset of $\mathbb{F}$ with $|C| = \ell + 1$, then
  \begin{align}\label{eq:lagrange1}
    h(X) = \sum_{i\in C}h(i)\delta_i(X)
  \end{align}
  where $\delta_i(X)$ is a degree $l$ polynomial s.t. for all $i,j \in C$, $\delta_i(j) = 0$ if $i \neq j$ and $\delta_i(i) = 1$ if $i = j$.
  \begin{align}\label{eq:lagrange2}
    \delta_i(X) = \prod_{j \in C,j\neq i} \frac{X-j}{i-j}
  \end{align}
\end{definition}

\begin{lemma}\label{lem:lagrange}
  Given any set of pairs $S = \{x_i, y_i \in F| i \in C\}$, $|C| = \ell + 1$, we can construct exactly one polynomial $h$ with $h(i) = y_i$ of degree at most $\ell$ as
  \begin{align*}
    h(X) = \sum_{i \in C} y_i \delta_i(X)
  \end{align*}
  \noindent and that all coefficients of this polynomial can be efficiently computed from $S$.
\end{lemma}

It from \autoref{def:lagrange} and \autoref{eq:lagrange1} that the secret can be easily \textit{reconstructed}. Furthermore, if there existed two solution polynomials $h, h'$ then $h(x) - h'(x)$ would be a non-zero polynomial of degree at most $\ell$ and would have at least $\ell + 1$ roots, which cannot exist. This gives use \textit{correctness}

For \textit{$(\ell)$-privacy}, consider an index set $I$ with $|I| \leq \ell$. For a sharing of a secret $s$ and the resulting polynomial $h(x)$, where $h(0) = s$, it evaluates to a set of shares $\{h(i) \mid i \in I\}$.
Conversely, consider a random set of shares $A = \{s_i \mid i \in I\}$. It follows from \autoref{lem:lagrange} that there exists exactly one polynomial $h_A(x)$ such that
\begin{align*}
  h_A(0) & = s,                                 \\
  h_A(i) & = s_i \quad \text{for all } i \in I.
\end{align*}
This implies that there are $|\mathbb{F}|^\ell$ possible polynomials which in turn implies that for any set of $\ell$ shares, the possible reconstructed secret $s$ is uniform in $\mathbb{F}$. See~\autoref{fig:shamir} for a visual example.

Furthermore, the $(\ell, n)$ the scheme is $(+, +)$-homomorphic. This follows from the fact that polynomials over $\mathbb{F}$ are \textit{distributive over addition}. In other words, given two secrets $a, b$ and their corresponding polynomials $h_a(x)$ and $h_b(x)$, we have
\begin{align*}
                & (h_a + h_b)(i) = h_a(i) + h_b(i)          \\
  \Rightarrow\  & (h_a + h_b)(0) = h_a(0) + h_b(0) = a + b.
\end{align*}

\begin{figure}
  \centering
  \includegraphics[width=0.8\textwidth]{images/shamir.png}
  \caption{Visualization of the Shamir's polynomial based secret sharing scheme. The dark line shows the specific sharing instance for a secret $s$ and a threshold $\ell=2$. With three shares, the secret can be reconstructed by interpolating the polynomial from the shares. However, with one or two shares, the reconstructed polynomial is not unique as we can see from the dotted lines.}\label{fig:shamir}
\end{figure}

\subsection{Secure Multi-Party Computation}\label{sec:mpc}

Secure Multi-Party Computation (MPC) refers to a cryptographic protocol enabling multiple parties to jointly compute a function $f$ represented by a circuit $C$, ensuring that no information about the individual inputs is revealed beyond what can be inferred from the output of $C$. In the following, sections we will denote a MPC protocol by $\Pi_f$ for an $n$-party functionality $f(x, w_1, \dots, w_n)$, a public input $x$ and secret inputs $w_i$ of the party $P_i$. The goal is to compute the function $f$ on the inputs of the $n$ parties while upholding the following properties~\cite{cramer2015secure}

\begin{definition}[Correctness]\label{def:mpc-correctness}
  We say that $\Pi$ realizes a deterministic $n$-party functionality $f(x, w_1, \dots, w_n)$ with perfect (resp., statistical) correctness if for all inputs $x, w_1, \dots, w_n$, the probability that the output of some player is different from the output of $f$ is $0$ (resp., negligible in $k$), where the probability is over the independent choices of the random inputs $r_1, \dots, r_n$.
\end{definition}

\begin{definition}[Privacy]\label{def:mpc-privacy}
  The protocol ensures that no party learns anything about the inputs of the other parties beyond what can be inferred from the output of the function.
\end{definition}

\begin{definition}[$\ell$-privacy]\label{def:mpc-ell-privacy}
  The MPC protocol has privacy up to $\ell$ malicious parties.\footnote{Most often achieved by using a $(\ell, n)$-private secret sharing scheme \autoref{def:mpc-ss-threshold}}
\end{definition}

In terms of security needed for the SD-in-the-Head protocol, we define the following security requirements for the MPC protocol $\Pi_f$

\begin{itemize}
  \item \textbf{Semi-honest security}: The protocol is secure against semi-honest adversaries, where parties follow the protocol but may attempt to learn information from the messages they receive.
  \item \textbf{Low-threshold security}: The protocol is secure against a coalition of up to $\ell$ parties, where $\ell$ is the threshold. This is also known as a $\ell$-private MPC protocol.
\end{itemize}

In order to prove the validity of a MPC protocol that have been run in the head, we define the following notion of \textit{consistency}~\cite{ishai2007zero}.

\begin{definition}[MPC view]\label{def:mpc-view}
  We denote the view $V_i$ of a party $P_i$ in the protocol as $(i, x, r_i, w_i, (m_1, \dots, m_j))$ where $r_i$ is the randomness used by $P_i$, $w_i$ is the secret share, and $(m_1, \dots, m_j)$ is the messages received by $P_i$ in the first $j$ rounds of the protocol. Note that the messages sent by the parties can be inferred from the $V_i$ by invoking $\Pi$.
\end{definition}

\begin{definition}[Consistent views]\label{def:mpc-consistent-view}
  We say that a pair of views $V_i, V_j$ are consistent (with respect to the protocol $\Pi$ and some public input $x$) if the outgoing messages implicit in $V_i, x$ are identical to the incoming messages reported in $V_j$ and vice versa.
\end{definition}

\begin{lemma}[Local vs. global consistency]\label{lem:consistency}
  Let $\Pi$ be an $n$-party protocol as above and $x$ be a public input.
  Let $V_1, \dots, V_n$ be an $n$-tuple of (possibly incorrect) views. Then all pairs of views $V_i, V_j$ are consistent with respect to $\Pi$ and $x$ if and only if there exists an honest execution of $\Pi$ with public input $x$ (and some choice of private inputs $w_i$ and random inputs $r_i$) in which $V_i$ is the view of $P_i$ for every $1 \leq i \leq n$.
\end{lemma}

\textbf{Proof of \autoref{lem:consistency}} The \textit{if} direction is trivial and follows from \autoref{def:mpc-view} and \autoref{def:mpc-consistent-view}.
The \textit{only if} direction is shown by the following. Consider $n$ pairwise consistent views $V_1, \dots, V_n$ \autoref{def:mpc-view}. Let us define an MPC protocol $\Pi$ by its \textit{next sent message} function $\Pi_j(i,x,w_i,r_i, (m_1, \dots, m_j)) = m_{j+1}$. The pairwise consistency, \autoref{def:mpc-consistent-view} and $\Pi_j$ implies, by induction that after $d$ rounds, that the actual view of all parties $P_i$ is the same as the view of $P_i$ in the first $d$ rounds. It follows that the views $V_i, \dots, V_n$ are consistent with the full execution of $\Pi$. \qed

\subsection{Zero-Knowledge Proofs}\label{sec:zk}
This section will give a brief introduction to \textit{two-party interactive} Zero-Knowledge (ZK) schemes for a \textit{prover} and a \textit{verifier}. The intuition behind a Zero-Knowledge proof of knowledge is that the prover can convince the verifier that they know a secret $w$, such that $x$ is true, without revealing the secret $w$ to the verifier. This could be someone wanting to prove that they are above the age of 18 without revealing their age. We denote a ZK $\Pi_{\mathcal{R}}$ for some NP relation $\mathcal{R}(x, w)$. Let $x$ be a public statement in \textbf{NP} and $w$ be a witness such that $(x, w) \in \mathcal{R}$~\cite{feneuil2023threshold}.
\todo{Use the ivan reference}

\begin{definition}
  Let $x$ be a statement of language $L$ in \textbf{NP}, and $W(x)$ the set of witnesses for $x$ such that the following relation holds:
  \begin{align*}
    \mathcal{R} = \{(x, w)\; x \in L, w \in W(x)\}
  \end{align*}
\end{definition}

\section{Syndrome Decoding Problem}\label{sec:syndrome}

The SD-in-the-Head protocol is built on the computational hardness of the Syndrome Decoding (SD) problem for random linear codes over a finite field (see \autoref{sec:gf256}). This protocol uses a variant of the SD problem, referred to as the \textit{coset weights} problem, first introduced by Berlekamp and McEliece in 1978~\cite{berlekamp1978inherent}. The problem is defined as follows:
\begin{definition}\label{def:syndrome}
  Given $H \in \mathbb{F}^{(m-k)\times m}_q$ and $y \in \mathbb{F}^{m-k}_q$. The problem is to find $x \in \mathbb{F}^m_q$ s.t.\ wt$(x) \leq w$ such that $Hx = y$.
\end{definition}
Generating such an instance is straightforward: one can construct a uniformly random parity-check matrix $H$ and a codeword $x$ (with $wt(x) \leq w$), and then compute the syndrome $y = Hx$. In the SD-in-the-Head protocol, the values of the matrix $H$ and the syndrome $y$ are elements of the finite field $\mathbb{F}_q$ explained previously in \autoref{sec:gf256}. The SD problem is well-known to be NP-complete for random instances~\cite{berlekamp1978inherent} also referred to as the general decoding problem.To illustrate the computational difficulty, solving the problem using brute force would require $O(\binom{m}{w} q^w)$ operations, which is computationally infeasible for large $m$ and $k$.

\subsubsection{Standard form of the parity-check matrix}\label{sec:standard_form_of_the_parity_check_matrix}
To improve the performance and reduce the key size of the protocol, its possible to utilize the fact that the matrix $H$ can be in standard form. $H = (H'|I_{m-k}) $ Where $H' \in \mathbb{F}^{(m-k)\times k}_q$, and $I_{m-k}$ is the identity matrix of size $m-k$. This allows for the following representation of the syndrome:
\begin{equation}
  y = Hx = H'x_a + x_b\label{eq:standard_form_of_the_parity_check_matrix}
\end{equation}
with $x = (x_a | x_b)$. This improves the performance of the algorithms used in the SD-in-the-Head protocol in the following ways
\begin{itemize}
  \item At the MPC layer, we only need to reveal one share $x_a$. Due to the fact that the other share $x_b$ can simply be recomputed by $x_b = y - H'x_a$.
  \item By linearity of the above relation one only needs to send $x_a$ in order to recover $Hx = H'x_a + x_b$ the Syndrome decoding instance. So from a sharing of $x_a$ one can check the correctness of the SD instance.
\end{itemize}

\subsubsection{Polynomial representation of SD}\label{sec:polynomial_representation}
In order to convert the SD problem into a form that allows for \textit{Multi Party Computation in the Head} (MPCitH, which we will exlpain further in \autoref{sec:mpcinth}) the SD-in-the-Head protocol is based on three (witness-dependent) polynomials $S, Q$ and $P$, and one public polynomial $F$.
These are used for checking the correctness of the SD solution by verifying the following relation:
\begin{equation}
  \centering
  S\cdot Q = P\cdot F\label{eq:polynomial_representation}
\end{equation}
Let $f_1,\dots, f_q$ denote all the elements of $\mathbb{F}_q$ and $x\in \mathbb{F}^m_q$ is a binary vector with hamming weight $wt(x) /leq w$, then the polynomials are defined as:
\begin{itemize}
  \item $S\in \mathbb{F}_q[X]$ is the Lagrange interpolation of the coordinates of $x$, such that it matches $S(f_i) = x_i$ for $i\in [1:m]$ and has degree $\text{deg}(S) \leq m-1$
  \item $Q\in \mathbb{F}_q[X]$ is defined by $Q(X) = \prod_{i\in E}(X - f_i)$. $E \subset [1:m]$ with order $|E| = w$, such that $E$ contains the non-zero coordinates of $x$. $Q$ has degree $\text{deg}(Q) = w$.
  \item $P\in \mathbb{F}_q[X]$ is defined as $P = S\cdot Q/F$ and has degree $\text{deg}(P) \leq w-1$. By definition the polynomial $F$ divides $S\cdot Q$.
  \item $F\in \mathbb{F}_q[X]$ is the \textit{vanishing polynomial} of the set ${f_1, \dots, f_m}$ also defined as $F(X) = \prod_{i\in [1:m]}(X - f_i)$ and has degree $\text{deg}(F) = m$.
\end{itemize}
We can now look at the relation in \autoref{eq:polynomial_representation}. If we look at the left-hand side, which has the following property by design $S\cdot Q(f_i) = 0 \ \forall\ f_i \in [1:m]$. This comes from the fact that the polynomial $S(f_i) = 0$ whenever $x_i = 0$, as it is the lagrange interpolation of $x$. Furthermore, the polynomial $Q(f_i)$ is zero whenever $f_i$ is a non-zero coordinate of $x$, which follows from the definition of $Q$.

For the right-hand side, the polynomial $F$ is the vanishing polynomial for the set ${f_1, \dots, f_m}$, so $F(f_i) = 0\ \forall\ f_i \in [1:m]$. The polynomial $P$ is needed to match the degree of $S \cdot Q$. As the degree of $F$ is $m \leq \text{deg}(S\cdot Q) \leq m + w - 1$.

It is now apparent that if the prover can convince the verifier that they know of polynomials $P,Q$ such that $S\cdot Q = F \cdot P = 0$ at all points $f_i \in [1:m]$. The following must hold, either $S(f_i) = x_i = 0$ or $Q(f_i) = 0$. However, the polynomial $Q$ can be zero in at most $w$ points based on the degree, this means that S is non-zero in at most $w$ points, based on the construction, which in turn implies that $x$ has weight at most $w$.

With this, we can define the soundness of the language for ZKP as follows:
\begin{equation}
  wt(x) \leq w \Leftrightarrow \exists P,Q \text{  with  }\text{deg}(P)\leq w-1\text{  and  }\text{deg}(Q) = w\text{ s.t. \autoref{eq:polynomial_representation} holds}\label{eq:soundness}
\end{equation}

We can now share a witness $(x_a, Q, P)$. Now based on the relation from \autoref{eq:standard_form_of_the_parity_check_matrix}, $x$ can be locally computed along with the polynomial $S$, this can then be used to run an equality test for the relation $S \cdot Q = P \cdot F$.

\subsubsection{False positive probability}\label{sub:equality_test}
The equality test from \autoref{eq:polynomial_representation} has a small probability of false positives, denoted $p$. To reduce this probability, the relation is evaluated at random points $\{r_k \in \mathbb{F}_q\}_{k\in[t]}$. By Schwartz-Zippel \autoref{lem:schwartz}, the probability of a false positive is bounded by $p \leq \frac{t}{q}$, where $q$ is the size of the finite field $\mathbb{F}_q$. In short, this makes it unlikely that the relation will hold for all points $r_k$ if the relation is not sound according to \autoref{eq:soundness}. Furthermore, we can tweak the parameters $t$ and $q$ to reduce $p$.

\begin{lemma}[Schwartz-Zippel]\label{lem:schwartz}
  For a non-zero polynomial $P \in \mathbb{S}[X]$ of degree $d \geq 0$. Let $\mathbb{R}$ be a finite subset of $\mathbb{S}$ and set of random points $[r_1, \dots, r_n] \in \mathbb{R}$, the probability that $\Pr[P(r_1, \dots, r_n) = 0] \leq d/|\mathbb{R}|$.
\end{lemma}

\section{MPC-in-the-Head}\label{sec:mpcinth}

The SD-in-the-Head protocol construction is based on the \textit{Multi-Party Computation in the Head} (MPCitH) framework. In this section we will give an introduction to the framework and how it can be used to construct ZK proofs, which in turn can be combined with the Fiat-Shamir heuristic to create a signature scheme.

The MPC-in-the-Head (MPCitH) framework, introduced by~\cite{ishai2007zero}, builds upon these techniques to construct generic zero-knowledge protocols (ZK). A ZK protocol allows a \textit{prover} to convince a \textit{verifier} of the validity of a statement without revealing any information about the inputs to the statement. The framework provides a versatile method of constructing protocols that are quantum safe as its security relies on assumptions that are still believed to be quantum secure. Namely, commitment schemes and hash functions~\cite{feneuil2023threshold} which have no known quantum algorithms that break their security.

We will give insight into the basic construction suggested by Ishai et al~\cite{ishai2007zero}.

\begin{definition}\label{def:mpcinth_basic}
  Given a semi-honest $\ell$-private MPC protocol $\Pi_f$ with perfect correctness, a relation $\mathcal{R}$ for some public statement $x$, a witness $w$. Let $w_i$ be an additive secret share of $\sh{w}$ for the party $P_i$. Let $f$ be a $n$-party functionality $f(x, w_1, \dots, w_n) = \mathcal{R}(x, w)$, i.e. $f(x,w)$ accepts if $(x,w) \in \mathcal{R}$.

  \begin{enumerate}[parsep=2pt, itemsep=0pt]
    \item The prover builds a random sharing of $\sh{w} = w_1, \dots, w_n$. Then
          \begin{enumerate}[nolistsep]
            \item Simulates the outputs of the MPC protocol $\Pi_f$ on the inputs $(x, w_1, \dots, w_n)$ and the randomness $r_1, \dots, r_n$.
            \item Prepares views $V_1, \dots, V_n$ of the parties in the protocol $\Pi_f$.\footnote{Remember that the views include the inputs, randomness and messages received by the parties.}
            \item Commits to each view $V_i$ using a secure commitment scheme, and sends ($\text{Commit}(V_1), \dots, \text{Commit}(V_n)$) to the verifier.
          \end{enumerate}
    \item The verifier picks $\ell$ random distinct indices $i \in [n]$ and sends them to the prover.
    \item The prover opens the corresponding $\ell$ commitments into the views $V_i$ and sends the openings to the verifier.
    \item The verifier accepts if and only if:
          \begin{enumerate}[nolistsep]
            \item\label{prop:mpcinth_commit} The views are valid according to the commitment scheme.
            \item\label{prop:mpcinth_consistent} The views are consistent according to the public input $x$.
            \item\label{prop:mpcinth_knowledge} The views output $1$ according to $\mathcal{R}$ meaning that $\mathcal{R}(x,w) = 1$.
          \end{enumerate}
  \end{enumerate}
\end{definition}

We see the following properties of the protocol:

\begin{lemma}[Completeness~\cite{ishai2007zero}]\label{def:mpcinth_completeness}
  A boolean function is said to be complete if it depends on all inputs of the function. Given an honest prover, $\mathcal{R}(x,w) = 1$ and the correctness of $\Pi_f$, all outputs of $P_i$ are one and all views are consistent.
\end{lemma}

\begin{lemma}[Soundness~\cite{ishai2007zero}]
  If the statement $x$ is false, i.e. $x \notin L$, then $\mathcal{R}(x,w) = 0$ for all $w$. By the correctness of $\Pi_f$, the output of all parties must be $0$. If the verifier accepts, then the prover must have created $\ell$ inconsistent views. This happens with probability at most $p = 1 / \binom{n}{\ell}$. The error probability can be reduced to $2^{-k}$ by repeating the protocol $O(kn^2)$ times.
\end{lemma}

\begin{lemma}[Zero-Knowledge~\cite{ishai2007zero}]
  The verifier only sees $\ell$ views, and therefore from the definition of $\Pi_f$ learns nothing of the secret witness $w$.
\end{lemma}

The basic MPC-in-the-Head protocol serves as a foundation for constructing zero-knowledge (ZK) proofs for any NP relation and has been demonstrated to yield relatively efficient ZK protocols~\cite{feneuil2023threshold,baum2020concretely,katz2018improved}. Giacomelli et al. and Chase et al.~\cite{katz2018improved,giacomelli2016zkboo,chase2017post} provided concrete implementations of the \emph{MPC-in-the-Head} approach and observed that employing a 3-party protocol $\Pi$ achieved optimal performance within the space of protocols they analyzed. However, due to the small number of parties, the soundness of the resulting honest-verifier zero-knowledge (HVZK) proof is relatively weak. As a consequence, a large number of parallel repetitions is required to achieve negligible soundness error. This increases the size of the proofs and the communication overhead of the protocols. However, recent work by Baum and Nof~\cite{baum2020concretely} and Feneuil and Rivain~\cite{feneuil2023threshold} propose promising approaches to mitigate these issues.

\subsection{MPC preprocessing}

Pre-processing in MPC is an optimization technique that leverages shares of correlated randomness, such as randomness derived from a shared seed, which remains independent of the inputs to the protocol. This independence allows the parties to accelerate computation and perform pre-computation in advance. Consequently, this approach enables the design of MPCitH protocols that can utilize any $N$-party MPC protocol during pre-processing with minimal impact on the proof size. This method was first introduced by~\cite{katz2018improved} as part of their \textit{cut-and-choose} approach.

\subsubsection{MPC over arithmetic circuits}\label{sec:zk-arithmetic-mpc-circuits}

In order to form a ZK proof for the SD polynomial relation in \autoref{sec:polynomial_representation}, we need a MPC protocol that allows us to securely compute an arithmetic circuit. For this we use \textit{beaver triples}.

\begin{definition}[Beaver triple~\cite{beaver1992efficient}]\label{def:beaver}
  A beaver triple is a tuple $(a, b, c)$ where $a \cdot b = c$ and $a, b, c \in \mathbb{F}_q$.
\end{definition}

Consider the protocol for a single multiplication gate below by~\cite{baum2020concretely}, based on the MPC protocol described in~\cite{damgaard2012multiparty}. Note, that a similar technique can be applied to square relations. However, we focus here solely on multiplication, as it suffices for the SD relation.

\begin{definition}[Verification of a multiplication triple by sacrificing another]\label{def:sacrifice}
  Given an input triple $(x,y,z) \in \mathbb{F}$ random shared triple $(\sh{a}, \sh{b}, \sh{c}) \in \mathbb{F}$, it is possible to verify the correctness of the statement $z = x \cdot y$ without revealing any information on either of the input. Define \texttt{open} according to \autoref{def:ss-opening}.
  \begin{enumerate}
    \item The parties generate a random $\epsilon \in \mathbb{F}$.
    \item The parties locally set $\sh{\alpha} = \epsilon\sh{x} + \sh{a}, \sh{\beta} = \sh{y} + \sh{b}$.
    \item The parties run \texttt{open}$(\sh{\alpha})$ and \texttt{open}$(\sh{\beta})$ to obtain $\alpha$ and $\beta$.
    \item The parties locally set $\sh{v} = \epsilon\sh{z} - \sh{c} + \alpha  \cdot \sh{b} + \beta  \cdot \sh{a} - \alpha  \cdot \beta$.
    \item The parties run \texttt{open}$(\sh{v})$ to obtain $v$ and accept iff $v = 0$.
  \end{enumerate}
\end{definition}
Observe that if both triples are correct multiplication triples (i.e., $z = xy$ and $c = ab$) then the parties will always accept since
\begin{align}
  v & = \epsilon \cdot z - c + \alpha \cdot b + \beta \cdot a - \alpha \cdot \beta                                            \\
    & = \epsilon \cdot xy - ab + (\epsilon \cdot x + a)b + (y + b)a - (\epsilon \cdot x + a)(y + b)                           \\
    & = \epsilon \cdot xy - ab + \epsilon \cdot xb + ab + ya + ba - \epsilon \cdot xy - \epsilon \cdot xb - ay - ab           \\
    & = (\epsilon \cdot xy - \epsilon \cdot xy) + (ab - ab) + (\epsilon \cdot xb - \epsilon \cdot xb) + (ya - ay) + (ba - ab) \\
    & = 0
\end{align}
Otherwise, the parties will accept with probability $\frac{1}{|\mathbb{F}|}$ following from \autoref{lem:mpc-multiplication}.

\begin{lemma}\label{lem:mpc-multiplication}
  If $(\sh{a}, \sh{b}, \sh{c})$ or $(\sh{x}, \sh{y}, \sh{z})$ is an incorrect multiplication triple then the parties output \texttt{Accept} in the sub-protocol above with probability $\frac{1}{|\mathbb{F}|}$.\footnote{A full proof of this can be found in~\cite{baum2020concretely}.}
\end{lemma}

\subsubsection{ZK from Cut-and-Choose}\label{sec:zk-cut-and-choose}

Building upon the MPC protocol above, \autoref{def:cut-and-choose} outlines the fundamental concept of the cut-and-choose technique. For a more comprehensive protocol, refer to the \textit{HVZKAoK Protocol using Cut and Choose} discussed in~\cite{baum2020concretely}. Note

\begin{definition}[ZK from Cut-and-Choose (Multiplication)]\label{def:cut-and-choose}
  Denote the security parameters $M, N, \tau$ a secure hash function \autoref{sec:prelim_hash} and a \textit{Random Oracle-based commitment scheme}. Given a $(N-1)$-private MPC protocol $\Pi_C$ for a circuit $C$ and an ideal functionality $\mathcal{F}$. \textit{Note} that we assume that there is only one multiplication gate in the circuit, to align with our purposes.
  \begin{enumerate}[itemsep=0pt, parsep=0pt]
    \item\label{itm:cut-choose-r1} For each $e \in [M]$ and for each party $P_i \in [N]$:
    \begin{enumerate}[nolistsep]
      \item $\mathcal{P}$ chooses a random $\texttt{salt} \leftarrow \{0,1\}^\lambda$ used for commitments.
      \item $\mathcal{P}$ chooses a master seed $sd_e$ and  expands $sd_e$ to $sd_{e,i}$ for each $i \in [N]$ using $\Pi_C$.\footnote{If the seed is expanded in a correlated way, you can send just $sd_e$ and have the verifer recreate the expanded seeds accordingly}
      \item $\mathcal{P}$ computes valid shares of beaver triples (\autoref{def:beaver}) $a_{e,i}$, $b_{e, i}$ and $c_{e,i}$ using $sd_{e,i}$.
            \begin{enumerate}
              \item To fix the initial randomness of $\sh{c_e}$, recompute $a_e$ and $b_e$ from the shares and set $c_e = a_e \cdot b_e$.
              \item Compute $\Delta_e = c_e - \sum_{i=1}^N c_{e,i}$.
              \item Set the triple of the multiplication gate to ($\sh{a_e}, \sh{b_e}, \sh{c_e} + \Delta_e$).
            \end{enumerate}
      \item $\mathcal{P}$ generates random sharings $w_i$ of the input $w$ using $sd_{e,i}$.
    \end{enumerate}
    \item $\mathcal{P}$ commits to the triples (including $\Delta_e$) and a sharing of the input $w$ along with the seeds $sd_{e,i}$\footnote{In order to further reduce communication all commitments can be hashed together to one hash.}
    \item $\mathcal{V}$ chooses a random challenge $E \subset [M]$ such that $|E| = \tau$ and sends it to $\mathcal{P}$. Set $\bar{E} = [M] \ E$
          \item\label{itm:cut-choose-r3} $\mathcal{P}$ sends the seeds $sd_e$ to $\mathcal{V}$ who verifies that the triples are correctly generated per step~\ref{itm:cut-choose-r1}
    \item $\mathcal{P}$ runs the MPC protocol $\Pi_C$ on the remaining $M-\tau$ pre-computed triples with different randomness $r_e$ resulting in shares $\alpha_e, \beta_e$.
    \item $\mathcal{P}$ commits to the views (\autoref{def:mpc-view}) of $\Pi_C$.
    \item For each $e \in \bar{E}$, $\mathcal{V}$ chooses a random $i \in [N]$ and sends it to $\mathcal{P}$.
    \item For each $e \in \bar{E}$: Let $I_e = [N] \setminus \{i_e\}$, $\mathcal{P}$ opens the commitments of views of the MPC protocol for $i \in I_e$ and sends the openings to $\mathcal{V}$ along with the seed $sd_{\bar{E}}$ and \texttt{salt} to recompute randomness.
    \item $\mathcal{V}$ outputs \texttt{Accept} if
          \begin{enumerate}[nolistsep]
            \item Step~\ref{itm:cut-choose-r3} was verified
            \item $\mathcal{V}$ verifies the commitments and consistency \autoref{def:mpc-consistent-view} of the views by running $\Pi_C$ as an honest participant using the pseudo-randomness from the seed and \texttt{salt}.
            \item $\mathcal{V}$ verifies for each $e \in \bar{E}$ that the recomputed output $\sh{y}$ of $\Pi_C$ is $y$
          \end{enumerate}
  \end{enumerate}
\end{definition}

Zero-Knowledge of the protocol is satisfied by the $N$-privacy of $\Pi_C$ (\autoref{def:mpc-ell-privacy}) as only $N-1$ views are shared, the witness $w$ is never revealed to the verifier.

We bound probability of $\mathcal{V}$ accepting a cheating outcome of \autoref{def:cut-and-choose}, i.e. that $C(w) \neq y$. First, we assume that $\mathcal{P}$ cheats in step~\ref{itm:cut-choose-r1} by incorrectly generating the precomputed values. The first challenge $E$ allows $\mathcal{V}$ to test $\tau$ triples. It's trivial to detect inconsistencies in the triples through the seed $sd_e$ by recomputing the triples and comparing them to the opened commitments. To reduce communication even more, the resulting computation from $\mathcal{P}$ can be hashed into a single hash which $\mathcal{V}$ tries to recompute. The probability that cheating is not detected is $\frac{\binom{M-c}{\tau}}{\binom{M}{tau}}$. Next, we assume that $\mathcal{P}$ cheats by deviating from $\Pi_C$. In order to make the output of the protocol $y$, $\mathcal{P}$ must deviate in $M-\tau-c$ emulations. As $N-1$ views are opened and verified, $\mathcal{P}$ can only cheat in on view of \textbf{one} party. The probability that this is not detected is $\frac{1}{N^{M-\tau-c}}$.

\begin{lemma}[Soundness of the cut-and-choose approach]\label{lem:cut-and-choose-soundness}
  Let $c$ be the number of pre-processing emulations where $\mathcal{P}$ cheats. The probability that $\mathcal{V}$ accepts the outcome of \autoref{def:cut-and-choose} for $C(w) \neq y$ is bounded by
  \begin{align*}
    \xi_c(M, N, \tau) = \max_{0 \leq c \leq M-\tau} \left\{\frac{\binom{M-c}{\tau}}{\binom{M}{tau} \cdot N^{M-\tau-c}} \right\}
  \end{align*}
\end{lemma}

To re-iterate, we set out to make it so that the communication complexity of the ZK MPCitH protocol was not bounded by $N$.

\todo{Explain how C\&C gives us the smaller proof in MPCitH}

\subsubsection{ZK from sacrificing}\label{sec:zk-sacrifice}

\todo{Explain changes from C\&C to this from Baum how it increases soundness for smaller $n$}

\subsection{Linear Secret Sharing Schemes (LSSS)}\label{sec:lsss}

\todo{Go through Feneuilis threshhold paper and desribe their gains from using LSSS}
\todo{Move specification of LSSS to SSS section}

\textit{Secret sharing schemes} (SSS) are a type of cryptographic protocol that allows for the distribution of a secret amongst a group of participants. The secret can only be reconstructed when a sufficient number of shares are combined together. The threshold variant of the SD-in-the-Head protocol relies on a low-threshold linear secret sharing scheme (\textit{LSSS}). Threshold secret sharing schemes allow for the reconstruction of a secret from a subset of shares of length $\ell$, where $\ell$ is the threshold. The threshold allows for the SD-in-the-Head protocol to be more communication efficient, as the amount of shares needed to reconstruct the secret is low.

\begin{definition}\label{def:sss}
  \textit{$S$ is a $(\ell,n)$ threshold SSS if it satisfies the following properties}:

  \begin{itemize}
    \item \textbf{Share generation}: Given a secret $s$, the scheme generates seeds $sd_{e,i}$ for each $i \in [N]$ using $\Pi_C$. $n$ shares $\texttt{share}(s) = \sh{s} = [s_1, s_2, \dots, s_n]$.
    \item \textbf{Reconstruction}: Given a subset of shares $\sh{s'}$ of size $\ell$, the scheme can reconstruct the secret $s = \texttt{open}(\sh{s'})$.
  \end{itemize}

\end{definition}

\textit{Linear secret charing schemes}, or $(+,+)$-homomorphic schemes, refers to secret sharing schemes that are linearly homomorphic over some field $\mathbb{F}$ (say the galois field \texttt{GF256} described in \autoref{sec:gf256}). This means that given shares $\sh{a}$ and $\sh{b}$ we have that
\begin{definition}
  A $(\ell,n)$ threshold SSS is $(+,+)$-homomorphic if for any two secrets $s_1$ and $s_2$ and their shares $[[s_1]]$ and $[[s_2]]$, the sum of the shares $[[s_1]] + [[s_2]] = [[s_{11} + s_{21}, s_{12} + s_{22}, \dots, s_{1n} + s_{2n}]]$ is equal to the share of the sum of the secrets $[[s_1 + s_2]]$ for the same subset of shares.
\end{definition}

\section{Fiat-Shamir Heuristic}\label{sec:fiatshamir}
To transform any zero-knowledge protocol into a signature scheme, one can use the approach described in~\cite{fiat1986prove}. Here, we outline the general concept.

Zero-knowledge protocols typically rely on the verifier to issue a challenge to the prover. This challenge serves as a source of randomness that the prover is not supposed to control. In the Fiat-Shamir framework, the original protocol is based on the problem of factoring integers -- a problem that is now vulnerable to quantum attacks, specifically Shor's algorithm. However, the fundamental method of converting a zero-knowledge protocol into a signature scheme remains unaffected.

To adapt a zero-knowledge protocol into a signature scheme, the prover eliminates the need for a verifier to provide randomness. Instead, the prover generates seeds $sd_{e,i}$ for each $i \in [N]$ using $\Pi_C$.the randomness using a pseudo-random function. This function takes as input the message to be signed along with some random values generated by the prover. The output of this function serves as the challenge. With this, the prover can compute the required values to prove authenticity without external interaction.

In the context of the SD-in-the-Head protocol, the prover uses the pseudo-randomly generated challenge to locally compute the witness values required for the proof. These witness values are derived through the MPC protocol, where the prover simulates the necessary computations internally. This eliminates the need for the verifier's active participation in the MPC protocol, streamlining the signature generation process.

%%%%%%%%%%%%%%%%%%%%%%%%%%%%%%%%%%%%%%%%%%%%%%%%%%%%%%%%%%%%%%%%%%%%%%%

\chapter{Specification}\label{ch:spec}

\todo{more detailed description of the algorithm
  e.g. how we sampled I[e] witness challenge
  table of spec params (with our code naming and different categories)}

For our implementation, we selected the \textit{threshold} variant of the SD-in-the-Head protocol~\cite{aguilarsyndrome11,feneuil2023threshold}, as this variant offers the most significant performance improvements compared to the initial protocol~\cite{feneuil2022syndrome} and the \textit{hypercube} variant~\cite{aguilarsyndrome11,aguilar2023return}, albeit at the cost of a slightly larger signature size.

In this chapter, we provide a specification of the initial SD-in-the-Head protocol, along with the details specific to the threshold variant. We include pseudo-code for the major subroutines and algorithms, as well as a comprehensive description of the parameters used in the protocol. Finally, we present an overview of the associated security guarantees.


\section{SD-in-the-Head Protocol}
The original SD-in-the-Head protocol is based on the computational hardness of the Syndrome Decoding (SD) problem for random linear codes over a finite field. As a reminder, problem is expressed as follows (see \autoref{sec:syndrome})
\begin{quote}
  For a parity check matrix $H \in \mathbb{F}_q^{(m-k)\times m}$ in standard form and a syndrome $y$. The challenge is to find a vector $x$ s.t. $Hx = y$ and $wt(x) \leq w$.
  \begin{align*}
    y = Hx = H'x_a + x_b \text{ \ \ \ where \ \ \ } H' \in \mathbb{F}_q^{(m-k)\times k}
  \end{align*}
  for $x = (x_a | x_b)$. We have the polynomial representation of the problem (see \autoref{sec:polynomial_representation})
  \begin{align*}
    S\cdot Q = P\cdot F \text{ \ \ \ where \ \ \ } S, Q, P, F \in \mathbb{F}_q[X]
  \end{align*}
\end{quote}
\noindent Testing that the relation holds has a small probability of a \textit{false positive}. We denote this as $p$. The protocol takes several steps to reduce $p$. First, the relation is evaluated at random points $\{r_k \in \mathbb{F}_q\}_{k\in[t]}$. Second, the field of the random points is extended from $\mathbb{F}_q$ to $\mathbb{F}_{q^4}$.

security

The SD-in-the-Head protocol is based on the \textit{Multi-Party Computation in the Head} (MPCitH) framework (see \autoref{sec:mpcinth}). \todo{Explain why we then use MPC and the linearity of polynomial evaluation with sacrificing random beaver triples}

\section{Field implementation}

\subsection{Field extension}

First extend $F_q$ to $F_{q^4} = F_q[Z] / (Z^2 + Z + 32(X))$. This is done by representing elements in the field as polynomials of degree at most 1 with coefficients in $F_q$. \todo{Define addition and multiplication}

\subsubsection{Polynomial evaluation}
\todo{Should this be moved to the specification section?}
For polynomial evaluation in the algorithm, seen in \autoref{sec:mpc}, we need to evaluate a polynomial $P(x)$ at a point $r \in F_q^\eta$. This is done by evaluating the polynomial at each coefficient and summing the results. The polynomial $P(x)$ is defined as
\begin{align}
  \textstyle\bigcup_{|Q|}(F_q)^{|Q|} \times F_q^\eta & \rightarrow F_q^\eta                 \nonumber                                                    \\
  Q(r)                                               & = \textstyle\sum_{i=0}^{|Q|} Q_i \cdot r^{i-1} &  & Q_i \in F_q, r \in F_q^\eta\label{eq:mpcpoly}
\end{align}

\todo{See gf256\_ext.rs}

\section{Merkle Tree Commitment}

\todo{Right now this section sounds more like a section in "Implementation". Instead it should describe the spec like their paper does with the addition of the getRevealedNodes method.}

We chose to follow the definitions for the merkle tree provided in \cite{aguilarsyndrome11}. However, we added a separate method that calculates bottom up the revealed nodes based on the view opening challenges. This is used to know the length of the authentication path that is added to the signature and is needed when parsing the signature for verification. The algorithm uses the queue structure which is also mentioned in the spec, where the tree structure is a 1D array with the last elements being the leaf. First it checks for empty input. Then we initiate an empty list to hold the revealed nodes. Now in order to traverse the tree we will get the $heightIndex$ and $lastIndex$, which are the current layer of the tree, and the index of the 1D array of where the current layer stops. Lastly we need a queue to hold the indexes. Now in order to find the revealed nodes it starts by adding the indexes of the selected leaves to the queue. Then we go through the queue until the next index is $1$ which indicates that we are at the root. In the while loop we first get the current index, and then check if we have gone up a layer in the tree, and if the current node is a left child. If the current node is a left child and it is the last index in the layer, then we can just skip to the next node because we know it has no sibling. If it is not the last node we get the next node as a candidate, if the queue is not empty. If the current node is still the left child, and the next node in the queue is the other child of the same parent we do not need to add this as a revealed node as it is part of the known values. But if the next node is not the right child of the current node we add it as a revealed node. If the current node is not the left child we add the left child to the revealed nodes. And then we add the parent index to the queue. When we exit the while loop we have traversed the tree bottom up adding for each layer the neighbours of the selected leaves. The algorithm is described in pseudocode in \autoref{alg:get-revealed-nodes}.

\todo{change last sentence}

\begin{breakablealgorithm}
  \caption{Get Revealed Nodes for Selected Leaves in Merkle Tree}\label{alg:get-revealed-nodes}
  \begin{algorithmic}[1]
    \Require $selectedLeaves$: List of selected leaf indices
    \Ensure List of revealed node indices

    \Function{getRevealedNodes}{$selectedLeaves$}
    \If{$selectedLeaves$ is empty}
    \State \Return empty list
    \EndIf

    \State $revealedNodes \gets \text{empty list}$
    \State $(heightIndex,\; lastIndex) \gets (1 \ll treeHeight,\; treeNodes - 1)$
    \State $queue \gets \text{empty queue}$


    \ForAll{$leaf \in selectedLeaves$} \Comment{Add commitments to the queue}
    \State $val \gets (1 \ll treeHeight) + leaf$
    \If{$\text{queue.add}(val)$ fails}
    \State \textbf{panic} ``Could not add element to queue''
    \EndIf
    \EndFor

    \While{$queue.peek() \neq 1$} \Comment{Do until we reach the root}
    \State $index \gets queue.remove()$

    \If{$index < heightIndex$}
    \State $heightIndex \gets heightIndex \gg 1$
    \State $lastIndex \gets lastIndex \gg 1$
    \EndIf

    \State $isLeftChild \gets (index \; \% \; 2 == 0)$

    \If{$isLeftChild \land index == lastIndex$}
    \Comment{Node has no sibling}
    \State \textbf{continue}
    \Else
    \State $candidateIndex \gets 0$
    \If{\textbf{not} $queue.isEmpty()$}
    \State $candidateIndex \gets queue.peek()$
    \EndIf

    \If{$isLeftChild \land (candidateIndex == index + 1)$}
    \State $queue.remove()$
    \ElsIf{$isLeftChild$}
    \State $revealedNodes.append(index + 1)$
    \Else
    \State $revealedNodes.append(index - 1)$
    \EndIf
    \EndIf

    \State $parentIndex \gets index \gg 1$
    \If{$queue.add(parentIndex)$ fails}
    \State \textbf{panic} ``Could not add element to queue''
    \EndIf
    \EndWhile

    \State \Return $revealedNodes$
    \EndFunction

  \end{algorithmic}
\end{breakablealgorithm}


\todo{Can we use Haraka v2}

\section{Hashing and XOF}

\todo{Write about Shake, Keccak} and how we initiate the hash function and XOF

\section{MPC computation}

The computation is based on HVZKAoK Protocol using imperfect preprocessing and sacrificing. Section 3.3~\cite{baum2020concretely}.

A toy example of the computation protocol computations, can be seen in~\autoref{fig:mpc}. We have the two computation methods. Here for 1 split and 1 evaluation point. This means that we have only value for each challenge and beaver triple. Note that any arithmetic is run in GF256, so addition and subtraction are both \texttt{XOR} and multiplication is modulus $x^8 + x^4 + x^3 + x + 1$. Furthermore, for negation we have that $-a = a$.

\begin{figure}
  \makebox[\textwidth]{
    \fbox{\begin{minipage}[t]{.45\textwidth}
        \noindent \texttt{PartyComputation}
        \begin{flalign*}
           & \textit{Input: }                                                               \\
           & (s_a, Q', P, a, b, c), (\overline{\alpha}, \overline{\beta}), (H', y)          \\
           & (\epsilon, r), \texttt{with\_offset}                                           \\
           & \textit{Output: }                                                              \\
           & (\alpha, \beta, v)                                                             \\\\
           & Q = Q'_1\text{ if \texttt{with\_offset} else }Q'_0                             \\
           & S = (s_a | y + H's_a) \text{ if \texttt{with\_offset} else } (s_a | H's_a)     \\
           & v = -c                                                                         \\
           & \alpha = \epsilon \cdot Q(r) + a                                               \\
           & \beta = S(r) + b                                                               \\
           & v \mathrel{{+}{=}} \epsilon \cdot F(r) \cdot P(r)                              \\
           & v \mathrel{{+}{=}} \overline{\alpha} \cdot b + \overline{\beta} \cdot a        \\
           & v \mathrel{{+}{=}} - \alpha \cdot \beta \text{ \ \ \ if \texttt{with\_offset}}
        \end{flalign*}
      \end{minipage}}
    \hfill
    \noindent
    \fbox{\begin{minipage}[t]{.45\textwidth}
        \noindent \texttt{InverseComputation}
        \begin{flalign*}
           & \textit{Input: }                                                                 \\
           & (s_a, Q', P), (\alpha, \beta, v), (\overline{\alpha}, \overline{\beta}), (H', y) \\
           & (\epsilon, r), \texttt{with\_offset}                                             \\
           & \textit{Output: }                                                                \\
           & (a, b, c)                                                                        \\\\
           & Q = Q_1\text{ if \texttt{with\_offset} else }Q_0                                 \\
           & S = (s_a | y + H's_a) \text{ if \texttt{with\_offset} else } (s_a | H's_a)       \\
           & c = -v                                                                           \\
           & a = \alpha - \epsilon \cdot Q(r)                                                 \\
           & b = \beta - S(r)                                                                 \\
           & c \mathrel{{+}{=}} \epsilon \cdot F(r) \cdot P(r)                                \\
           & c \mathrel{{+}{=}} \overline{\alpha} \cdot b + \overline{\beta} \cdot a          \\
           & c \mathrel{{+}{=}} - \alpha \cdot \beta \text{ \ \ \ if \texttt{with\_offset}}
        \end{flalign*}
      \end{minipage}}}
  \caption{Simplified version of the MPC party computation and inverse computation. $Q_0$ means that $Q$ is completed with a $0$ for leading coefficient. Furthermore, $F$ is precomputed. Note that all arithmetic is done in $\mathbb{F}_q = GF256$. All elements are in $\mathbb{F}_q^\eta$ except for the coefficients of $Q$, $S$ and $P$ which are in $\mathbb{F}_q$.}\label{fig:mpc}
\end{figure}
\bigskip

Note that the

If we first instantiate an input $i$ and one random input $i^*$ (like the \texttt{input\_coef}).Then the input share is generated by adding the two. Similar, but simpler, to the input share generation of Algorithm 12, line 13 of the specification.
\begin{align*}
  i                & = (s_a, Q, P, a,b,c)                                             \\
  i^*              & = ({s_a}^*, Q^*, P^*, a^*,b^*,c^*)                               \\
  \sh{i} = i + i^* & = ({s_a} + {s_a}^*, Q + Q^*, P + P^*, a + a^*, b + b^*, c + c^*) \\
                   & = (\sh{s_a}, \sh{Q}, \sh{P}, \sh{a}, \sh{b}, \sh{c})             \\
  \texttt{chal}    & = (\epsilon, r)                                                  \\
  \texttt{pk}      & = (H', y)
\end{align*}
We also compute the plain broadcast share of the input as per Algorithm 12, line 18. Note that $\overline{v}$ is computed to zero and therefore removed from the computation in the implementation.
\begin{align*}
  (\overline{\alpha},\ \overline{\beta})         & =
  \texttt{PartyComputation}(i,\ (\overline{\alpha}, \overline{\beta}),\ \texttt{chal},\ \texttt{pk},\ \texttt{true})                                                                                   \\\\
  \overline{\alpha}                         =    & \epsilon \cdot Q_1(r) + a                                                                                                                           \\
  \overline{\beta}                          =    & S_y(r)  + b                                                                                                                                         \\
  \overline{v}                            =      & -c + \epsilon \cdot F(r) \cdot P(r) + \overline{\alpha} \cdot b + \overline{\beta} \cdot a  - \overline{\alpha} \cdot \overline{\beta}              \\
  \overline{v}                            =      & -c + \epsilon \cdot F(r) \cdot P(r) + (\epsilon \cdot Q_1(r) + a) \cdot b + (S_y(r)  + b) \cdot a  - (\epsilon \cdot Q_1(r) + a) \cdot (S_y(r) + b) \\
  \overline{v}                            =      & -c + \epsilon \cdot F(r) \cdot P(r)                                                                                                                 \\
                                                 & + \epsilon \cdot Q_1(r) \cdot b + c + S_y(r) \cdot a + c                                                                                            \\
                                                 & - \epsilon \cdot Q_1(r) \cdot S_y(r) - \epsilon \cdot Q_1(r) \cdot b - a \cdot S_y(r) - c                                                           \\
  \overline{v}                                 = & \ 0
\end{align*}
We then compute a broadcast share from the randomness and the broadcast, as per Algorithm 12, line 21.
\begin{align*}
  (\alpha^*, \beta^*, v^*) & = \texttt{PartyComputation}(i^*, (\overline{\alpha},
  \overline{\beta}), \texttt{chal}, \texttt{pk}, \texttt{false})                  \\\\
  \alpha^*                 & = \epsilon \cdot {Q^*}_0(r) + a^*                    \\
  \beta^*                  & =  {S^*}_0(r) + b^*
\end{align*}
This broadcast share is sent to the verifier along with the truncated input share (removing the beaver triples). The verifier then needs to recompute the input share beaver triples using the \texttt{InverseComputation} function. First we add the input share to the broadcast share as per Algorithm 13, line 8.
\begin{align*}
  (\alpha', \beta', v') & = (\alpha^*, \beta^*, v^*) + (\overline{\alpha},\ \overline{\beta}, 0)
  = (\alpha^* + \overline{\alpha}, \beta^* + \overline{\beta}, v^* + 0)                          \\\\
  \alpha'               & = \epsilon \cdot {Q^*}_0(r) + a^* + \overline{\alpha}                  \\
                        & = \epsilon \cdot {Q^*}_0(r) + a^* + \epsilon \cdot Q_1(r) + a          \\
  \beta'                & = {S^*}_0(r) + b^* + \overline{\beta}                                  \\
                        & = {S^*}_0(r) + b^* + S_y(r) + b                                        \\
\end{align*}
Next, the verifier computes the inverse of the broadcast share to recompute $(\sh{a}, \sh{b}, \sh{c})$ using the \texttt{InverseComputation} function. This is done as per Algorithm 13, line 10.
\begin{align*}
  (a', b', c') & = \texttt{InverseComputation}(\sh{i}, (\alpha', \beta', v'),
  (\overline{\alpha}, \overline{\beta}), \texttt{chal}, \texttt{pk}, \texttt{true})                                                     \\\\
  a'           & = \alpha' - \epsilon \cdot {\sh{Q}}_1(r)                                                                               \\
               & = \epsilon \cdot {Q^*}_0(r) + a^* + \epsilon \cdot Q_1(r) + a - \epsilon \cdot {\sh{Q}}_1(r)                           \\
               & = \epsilon \cdot {Q^*}_0(r) - \epsilon \cdot {\sh{Q}}_1(r) + \epsilon \cdot Q_1(r) + \sh{a}                            \\
               & = \epsilon \cdot ({Q^*}_0(r) - {\sh{Q}}_1(r) +  Q_1(r)) + \sh{a}                                                       \\
               & = \epsilon \cdot (Q^*_0(r) +  Q^*_0(r)) + \sh{a}                                             &  & \autoref{eq:mpcpoly} \\
               & = \sh{a}                                                                                     &  & \autoref{eq:mpcpoly} \\
  b'           & = \beta' - {\sh{S}}_y(r)                                                                                               \\
               & = {S^*}_0(r) + b^* + S_y(r) + b - {\sh{S}}_y(r)                                                                        \\
               & = {S^*}_0(r) - {\sh{S}}_y(r) + S_y(r)  + \sh{b}                                                                        \\
               & = {S^*}_0(r) - S^*_0(r)  + \sh{b}                                                            &  & \autoref{eq:mpcpoly} \\
               & = \sh{b}                                                                                     &  & \autoref{eq:mpcpoly} \\
\end{align*}
\todo{Want to explain the above in a more detailed manner? Specifically ${Q^*}_0(r) - {\sh{Q}}_1(r) = Q_1(r)$}

\section{Security}

The security analysis of the SD-in-the-Head signature scheme is based on the proposed standardization requirements from the 2022 NIST call for proposals of non-lattice based signature schemes~\cite{nistcall}. Therefore, as a preliminary, we will describe the reasoning and requirements of the NIST standardization process.

With the development of new quantum algorithms and unknowns in the capacities of the future quantum computers, there remain large uncertainties in estimating the security of the algorithms. To combat these uncertainties, NIST proposed for the 2022 stadardization effort, to define the security of submissions in a range of five categories. Each, with an easy-to-analyze cryptographic primitive providing the lower bound for a variety of metrics deemed relevant to practical security. The SD-in-the-Head specification provides security parameters adhering to categories one, three and five.

\begin{definition}\label{def:nistsec}
  Any attack that breaks the relevant security definition must require computational resources comparable to or greater than those required for key search on a block cipher with a 128-bit (e.g. AES-128), 192-bit (e.g. AES-192) and 256-bit (e.g. AES-256) for categories one, three and five respectively.
\end{definition}

In terms of quantum security, the complexity and capability of quantum algorithms are measured in terms of quantum circuit size, i.e. the number of quantum gates in the quantum circuit. In order to estimate the quantum security of the signature protocols, circuit size can be compared to the resources required to break the security of \autoref{def:nistsec}. Therefore,
according to the proposal by NIST, the SD-in-the-Head specification provides security metrics in terms of quantum circuit depth to optimal key recovery for AES-128, AES-192 and AES-256 for categories one, three and five respectively. These are estimated to be $2^{143}$, $2^{207}$ and $2^{272}$ classical gates~\cite{nistcall}.

\subsection{Security Definition}

The SD-in-the-Head signature scheme upholds the \textbf{E}xistential \textbf{U}n\textbf{f}orgeability under \textbf{C}hosen \textbf{M}essage \textbf{A}ttack (EUF-CMA) security property for digital signature schemes as this is the type of attack that NIST will evaluate signature proposals~\cite{nistcall,aguilarsyndrome11}. EUF-CMA works as the following game:

\begin{enumerate}
  \item The challenger generates seeds $sd_{e,i}$ for each $i \in [N]$ using $\Pi_C$.a key pair $(pk, sk)$ and sends $pk$ to the adversary.
  \item The adversary is then allowed to query a signing oracle for signatures of chosen messages $(m_1, \dots, m_r)$ and receives valid signatures $(\sigma_1, \dots, \sigma_r)$. For the NIST evaluation it is assumed that the adversary can query the signing oracle for up to $2^{64}$ chosen messages according to the adversary's running time. However, there is no requirement on the timing of the queries.
  \item The adversary then outputs a pair $(m^*, \sigma^*)$. The adversary wins if the following holds
        \begin{enumerate}
          \item $m^*$ has not been queried to the signing oracle.
          \item The pair $(m^*, \sigma^*)$ is a valid signature for $m^*$ under $pk$.
        \end{enumerate}
\end{enumerate}

\begin{table}[h]\label{tab:secparam}
  \centering
  \def\arraystretch{1.5}%  1 is the default, change whatever you need
  \begin{tabular}{cccccccccccccc}
    \specialrule{.1em}{.05em}{.05em}
    \multicolumn{2}{c}{\textbf{NIST security}} &      & \multicolumn{5}{c}{\textbf{SD parameters}} &     & \multicolumn{5}{c}{\textbf{MPCitH Parameters}}                                                             \\ \cline{1-2} \cline{4-8} \cline{10-14}
    Category                                   & Bits &                                            & $q$ & $m$                                            & $k$ & $w$ & $d$ &  & N   & $\ell$ & $\tau$ & $\eta$ & $t$ \\ \hline
    \textbf{I}                                 & 143  & \textit{}                                  & 256 & 242                                            & 126 & 87  & 1   &  & $q$ & 3      & 17     & 4      & 7   \\
    \textbf{III}                               & 207  &                                            & 256 & 376                                            & 220 & 114 & 2   &  & $q$ & 3      & 26     & 4      & 10  \\
    \textbf{V}                                 & 272  &                                            & 256 & 494                                            & 282 & 156 & 2   &  & $q$ & 3      & 34     & 4      & 13  \\ \specialrule{.1em}{.05em}{.05em}
  \end{tabular}
  \caption{Security parameters for the SDitH protocol for categories one, three and five~\cite{aguilarsyndrome11}.}
\end{table}

\begin{table}[h]\label{tab:hashparam}
  \centering
  \def\arraystretch{1.5}%  1 is the default, change whatever you need
  \begin{tabular}{clll}
    \specialrule{.1em}{.05em}{.05em}
         & \multicolumn{1}{c}{\textbf{I}} & \multicolumn{1}{c}{\textbf{III}} & \multicolumn{1}{c}{\textbf{V}} \\ \cline{2-4}
    Hash & SHA3-256                       & SHA3-384                         & SHA3-512                       \\
    XOF  & SHAKE-128                      & SHAKE-256                        & SHAKE-256                      \\ \specialrule{.1em}{.05em}{.05em}
  \end{tabular}
  \caption{Hash and XOF functions used in the SDitH protocol for categories one, three and five~\cite{aguilarsyndrome11}.}
\end{table}

\subsection{Assumptions}\label{sec:assumptions}
The SD-in-the-Head protocol is secure under the following assumptions:

Syndrome Decoding instances cannot be solved in complexity lower than $2^\kappa$ corresponding to the complexity of breaking AES by exhaustive search (see \autoref{def:nistsec}) in terms of quantum circuit size. For this, $\kappa$ is defined as $143$, $207$ and $272$ for the categories~\cite{nistcall}. Furthermore, the XOF primitive used is secure with 128-bit, 192-bit, 256-bit security levels for each of the categories respectively. Finally, the Hash function used \textit{behaves as random oracle}. Specifically, security holds in Random Oracle Model (ROM) and Quantum Random Oracle Model (QROM).

\subsection{Security of the Syndrome Detection Problem}\label{sec:sdsec}

Recall the definition of the syndrome detection problem from \autoref{def:syndrome}. The hardness of the SD problem is well established and the coset variant used in the SD-in-the-Head signature scheme, has been shown to be \textit{NP-complete}~\cite{berlekamp1978inherent,aguilarsyndrome11}. Furthermore, a brute force attack of guessing $x$ would require finding a unique correct solution in $\binom{m}{w} q^w$ which is infeasible for the parameters outlined in the specification. As an example, for category one, the number of possible solutions are $\approx 7.7 \times 10^{276}$. There exists more sophisticated algorithms for solving the SD problem, such as the Generalized Birthday Algorithms (GBA) and Information Set Decoding (ISD)~\cite{prange1962use}.


%%%%%%%%%%%%%%%%%%%%%%%%%%%%%%%%%%%%%%%%%%%%%%%%%%%%%%%%%%%%%%%%%%%%%%%

\chapter{Implementation}\label{ch:impl}

We set out to implement the SD-in-the-Head Protocol using a modern, well-maintained language that provides the necessary performance and security for a post-quantum secure signature scheme. To achieve this, we selected Rust~\cite{rustlangRustProgramming,nistsaferlanguages,lurklurkEffectiveRust,rustlangPerformanceBook} as the programming language. As a principle and to give ourselves the ability to manage the protocol effectively, we aimed to develop as much of the project's code as possible independently, minimizing reliance on pre-existing packages. This approach included implementing subroutines for the Galois Finite Field and Merkle Tree Commitment Scheme. However, for cryptographic primitives such as hash functions~\cite{blakethree,tinykeccak}, we utilized external libraries. This reliance is common, as developing such primitives securely often requires an entire dedicated effort and a whole project in itself.

This section provides a detailed overview of our implementation. We begin with an introduction to the Rust programming language, explaining our rationale for choosing it and highlighting specific features that are relevant to the subsequent sections. Following this, we present a brief walkthrough of each module in our implementation, detailing how the code aligns with the SD-in-the-Head specification. Additionally, we describe the methods used to test and optimize these modules and discuss the challenges encountered during their development.

\todo{FOr each section of code add following: What is it for (reference to spec section. What issues did we run into. Which Rust specific patterns are we utilizing. How are we testing it. )}

code sections
code re-usability with traits for categories.

compiling constants for categories

can we use other hashes that still provide the same security assumption from \autoref{sec:assumptions} as post-quantum security (Xoodyak, KangarooTwelve, Haraka v2)


\section{The Rust Programming language}

Before delving into the specifics of our implementation, we need to answer the question: ``Why is Rust a good choice for our implementation?''.

In 2024, the government of the United States of America took a stance on the future of cyber secure programming languages~\cite{whitehouse2024memorysafe}. In their report, they underline the need for secure building blocks when developing secure software. They point to the fact that \textit{Common Weakness Enumeration} (CWE) data highlights \textit{memory safety vulnerabilities} (MSV) as one of the most pervasive classes of vulnerabilities. MSV's exploit how memory can be accessed, written, allocated, or deallocated in ways that are beyond the scope of the program. Common examples of programming languages that are vulnerable include C and C++, which are widely used due to their high performance. To prevent such vulnerabilities, the report emphasizes the importance of using programming languages that inherently provide memory safety and does not require the developers to manually ensure security.

The Rust programming language addresses memory safety through its unique \textit{borrow checker}~\autoref{sec:rustborrow}, which enforces strict rules on how memory is accessed and managed. In the same year as the White House's report, NIST also released their recommendations on \textit{Safer Languages}~\cite{nistsaferlanguages}, which highlighted Rust as a leading choice. Together, these endorsements underscore that Rust is well-suited for implementing a protocol that adheres to modern security standards.

In addition to its focus on safety, Rust benefits from an extensive library of community-maintained documentation and resources~\cite{rustlangRustProgramming,rustlangPerformanceBook,lurklurkEffectiveRust}. These resources provide invaluable support for developers, enabling them to better understand the language's features and effectively utilize its capabilities.

In this section we will give an overview of the key features of the Rust programming language that make it an ideal choice for this project as we aimed to build a robust by leveraging Rust's safety mechanisms, performance optimization capabilities, dynamic language features, and strong testing infrastructure.

\subsection{Borrow checker}\label{sec:rustborrow} % mut references
To ensure memory safety, Rust uses a borrow checker to prevent data races and memory leaks. We will now dive into the different aspects of the borrow checker and how it works in Rust.

\subsubsection{The Stack and Heap}
To understand how the borrow checker works in Rust, it is crucial to first grasp the distinction between the stack and the heap -- two memory management systems available in Rust~\cite[ch.4]{rustlangRustProgramming}, each with unique characteristics and use cases.

The \textbf{stack} stores values in a last-in-first-out (LIFO) order, meaning the last value added is the first to be removed. You can think of it like a stack of thermal detonators in a crate: the last detonator placed in the crate is the first one you grab when you need it. All data stored on the stack must have a fixed, known size at compile time.

The \textbf{heap}, in contrast, is more flexible. When you allocate memory for a value on the heap, you request a specific amount of memory, and the heap manager finds a suitable location for it. A helpful analogy for the heap is docking a spaceship on Coruscant: you tell the spaceport how large your ship is, and they assign you to a landing pad that fits your ship's dimensions, noting where you've been docked.

As you might have guessed, the stack is faster than the heap. This is because the stack is a fixed-size structure, while the heap is a variable-sized structure. However, the stack has a fixed size, which means that it can only store a limited number of values. If you try to allocate more memory than the stack can hold, you will get an error. The heap, on the other hand, can store an unlimited number of values, as it is able to dynamically allocate memory as needed.

\subsubsection{Ownership}
Now that we have given a brief introduction to the how the stack and the heap work, we can delve into the ownership system in Rust. The ownership system is a crucial aspect of Rust's memory management, as it ensures that memory is allocated and deallocated correctly. In Rust, every value has a unique owner, and there can only be one owner at a time. When a value is created, it is assigned to the current scope, and when the scope ends, the value is dropped. This process is known as \textbf{ownership transfer}.
A simple example of how the scope works is as follows:
\begin{minted}{rust}
{
  // d can not be accessed yet because it has not been assigned to the current scope
  let d = "Peace is a lie";

  // Now we can use d
 
} // d is dropped here
\end{minted}
In this example, we used a string literal to initialize the variable \texttt{d}. A string literal is a statically allocated string that is stored in the binary.
Now to illustrate the ownership, we need a data type that is more complex than a string literal. The string literal is of known size and is allocated on the stack, meaning that it can be easily copied on stack and popped when the current scope ends. This however means that we can not mutate the string literal by some user input. Instead, we will use a \texttt{String} type, which is a dynamically allocated string that can be mutated:
\begin{minted}{rust}
  let d = String::from("Peace is a lie");

  d.pust_str(", there is only passion.");
  
  println!("{d}"); // Will print `Peace is a lie, there is only passion.`
\end{minted}
Here the idea is that because a string literal is off a known size at compile time it can be allocated on the stack. Whereas the type \textit{String} is dynamically allocated we would have to predict every size string in order to allocate a large enough space on the stack. This is the reason it is allocated on the heap, and a \textit{pointer} is instead saved on the stack.
Now comes the problem of what happens in the following cases:
\begin{minted}{rust}
  // Case 1: Stack allocated types
  let d = "Through passion, I gain strength"
  let v = d;

  // Case 2: Heap allocated types
  let d = String::from("Through strength, I gain power");
  let v = d;
\end{minted}
\begin{itemize}
  \item Case 1: The string literal \texttt{d} has the \textit{copy} trait and is copied on the stack.
  \item Case 2: $d$ is now a pointer to the heap allocation of the String.
\end{itemize}

\subsubsection{Borrowing and references}

\subsubsection{Lifetime}

% Reusing collections: https://nnethercote.github.io/perf-book/heap-allocations.html#reusing-collections
% Bounds Checks: https://nnethercote.github.io/perf-book/bounds-checks.html
% Alternative allocators: https://nnethercote.github.io/perf-book/build-configuration.html#alternative-allocators
\subsection{Modules}
Rust provides a module system that allow you to organize your code into logical units. Making it it easier to understand and maintain the codebase. A module is a collection of related items, such as functions, structs, and constants, that are grouped together. Modules can be nested, allowing for a hierarchical structure of code.
We utilized these features to organize our logic effectively, such as encapsulating field arithmetic within a module and separating components into subfolders and subroutines related to the main signature algorithm.
We have outlined the structure of the code below:
\begin{lstlisting}[style=tree][h]
.
├── api.rs
├── arith
│   ├── gf256
│   │   ├── ...
│   │   └── mod.rs
│   └── mod.rs
├── mpc
│   ├── ...
│   └── mod.rs
├── signature
│   ├── ...
│   └── mod.rs
├── subroutines
│   ├── ...
│   └── mod.rs
└── utils
    ├── ...
    └── mod.rs
\end{lstlisting}
\subsubsection{Visibility} %pub(crate)
To restrict access to code within specific modules, we use the \textit{pub(crate)} visibility modifier. This ensures that the code is accessible only to modules within the same crate, helping to prevent unintended access to private or internal functionality.

Additionally, we selectively mark certain modules and functions with the \textit{pub} visibility modifier to export them for use in other crates. This approach is particularly useful when creating a library file that can be accessed by benchmarking files, as these files reside outside the main crate.

\subsection{Correctness (Types and Testing)}
When creating and iterating on software, it is crucial that you develop in a way that you can continually deliver functional and correct code. There are many patterns that help ensuring this. Rust provides several. in this section we will describe two -- Types and Automated tests.

\subsubsection{The Type System}
Every value in Rust has a \textit{data type}. Programs written in Rust are \textit{statically typed}, meaning the type of each value must be determined at compile time. The type informs the compiler about what operations the value supports and helps catch errors before they reach runtime~\cite[ch.3.2]{rustlangRustProgramming}. Rust's compiler often infers types automatically, but if it cannot, you must explicitly annotate the type of the value. Consider the following example:

\begin{minted}{rust} 
  let guess: u32 = "42".parse().expect("Not a number!"); 
\end{minted}

Here, the \texttt{parse} function requires the annotation of the variable \texttt{guess} to determine how the string should be parsed. Additionally, if you annotate \texttt{guess} with a type that does not implement the \texttt{parse} function for strings, the compiler will produce an error. This strict type system ensures that Rust minimizes the possibility of writing incorrect code -- at least in terms of the bounds of the language. While Rust enforces correctness in type usage, ensuring that a program operates semantically as intended, such as adhering to a protocol specification, often requires additional validation methods.

\subsubsection{Automated Testing}
Say that you are implementing an arithmetic field -- \textit{funny enough, we needed to to exactly that!} -- you would want your impletation to follow the properties that define the field. For example \textit{associativity over addition}
\begin{align}
  a + (b + c) = (a + b) + c
\end{align}
Such a property is hard to enforce using the type system. However, it is possible to enforce it using \textit{automated testing}. Consider the following examples
\begin{minted}{rust}
  fn gf256_add(a: u8, b: u8) -> u8 {
    a ^ b
  }

  #[cfg(test)]
  mod arithmetic_tests {
    use super::*
    
    #[test]
    fn test_add_associativity() {
      let a = 3;
      let b = 4;
      let c = 5;

      assert_eq!(
        gf256_add(gf256_add(a, b), c), 
        gf256_add(a, gf256_add(b, c))
      );
    }
  }
\end{minted}
This is a simple example of the features that Rust provides that to support automated testing. First we create an internal module \rust{mod arithmetic_tests}. Note the annotation, \rust{#[cfg(test)]}. This ensures that the module is only included when running tests and not in the final production code. 

Individual tests are made using the \mintinline{rust}|code|

\subsection{Traits}

\subsection{Constants} % Compile time

\subsection{Build script}

\subsection{Error handling with Result type} % https://www.lurklurk.org/effective-rust/panic.html

\subsection{Feature flags}

\subsection{Optimisation}
\subsubsection{Inlining}
\subsubsection{Parallelisation}
\subsubsection{SIMD}

\section{Subroutines}\label{sub:subroutines}
We started out by implementing the different subroutines that are used in the SD-in-the-Head protocol. We will describe the implementation of each subroutine in detail.

\subsection{PRG}
First we implemented a pseudo random generator (PRG) based on a extendable output hash function (XOF). We used a library called \textit{tiny-keccak}~\cite{tiny-keccak}, which contains implementations of Keccak derived functions specified in the NIST FIPS 202 standard, SP800-185 and KangarooTwelve.

Our initial implementation focused on the NIST category one, which requires a security level of 128 bits. For this we used the SHAKE-128 implementation. But when we later had to adhere to the NIST category tree and five, we had to be able to switch between different implementations of the underlying XOF. For this we implemented a generic trait for the XOF:
\inputminted[firstline=18, lastline=23]{rust}{../sdith/src/subroutines/prg/xof.rs}
This allows us to create a struct that will contain a generic XOF and allow for interchangeability between different XOF implementations:
\inputminted[firstline=25, lastline=27]{rust}{../sdith/src/subroutines/prg/xof.rs}
Now in order to use the correct XOF based on the security level needed for each category, we used a constant to swithc between the different implementations of the SHAKE XOF -- we will discuss how this constant was defined for each category in \autoref{sub:categories}:
\inputminted[firstline=37, lastline=43]{rust}{../sdith/src/subroutines/prg/xof.rs}
We could now call the squeeze funtion to get the random bytes necessary for the seed, salt and key generation.

\subsubsection{Hashing}\label{sec:hashing} % (fold)
As mentioned in the spec, we need a collision-resistant hash function. For this we also used the tiny-keccak library. And again due to the different security levels, we had to be able to switch between different implementations of the underlying hash function. These being the SHA3-256, SHA3-384 and SHA3-512.
This was done by creating a generic trait for the hash function:
\inputminted[firstline=11, lastline=17]{rust}{../sdith/src/subroutines/prg/hashing.rs}

% subsubsection Hashing (end)
%
\subsection{commitments}

\subsection{Merkle Tree impl}\label{sub:merkle_tree_impl}
Our merkle tree implementation was done using a sized array as data structure. Meaning that we save all the leafs of the tree in the last 256 values of the array.
\inputminted[firstline=18, lastline=23]{rust}{../sdith/src/subroutines/merkle_tree.rs}
We then build the tree buttom up by hashing the left and right children of the current node and storing the result in the next position in the array. We do this until we reach the root node, which we then return as our global commitment.


\subsubsection{Hashing, XOF, and Commitments}

\subsubsection{Galois field arithmetic}\label{sub:gf256_arith}
As stated earlier, we wanted to implement the underlying arithmetic our selves and not rely on there being a library for the arithmetic of the field. This led us to implement the galois field arithmetic.
We did this by creating a trait for field arithmetic:
\inputminted[firstline=10, lastline=66]{rust}{../sdith/src/arith/gf256/mod.rs}
This allows us to implement the field arithmetic for different types:
\begin{minted}{rust}
pub type FPoint = [u8; 4];

impl FieldArith for FPoint {

    fn field_mul(&self, rhs: Self) -> Self {
        gf256_ext32_mul(*self, rhs)
    }

}
\end{minted}
And we can now use the field arithmetic in our code to for example generate beaver triples:
\inputminted[firstline=152, lastline=152]{rust}{../sdith/src/mpc/beaver.rs}

Furthermore, all our arithmetic can be tested individually, as we can see in the following test:
\inputminted[firstline=68, lastline=96]{rust}{../sdith/src/arith/gf256/mod.rs}
\section{Key generation}

\section{MPC}\label{sub:mpc_algo}
After implementing all the subroutines, we could now do the MPC computations necessary for signing and verification.

\section{Comparison with spec impl}\label{sub:comparison_with_spec_impl}
In order to verify and debug our own code we tried to setup our implementation to fit with the specification implementation in C, meaning we had to run their code in order to get the intermediate data in order to compare the internal states.
Here we found a bug in the way the view-opening challenges was calculated, they had forgotten to finalize the Shake before squeezing, which was a problem because in the tiny keccak rust library this was always done when initializing a hash or xof.
Furthermore we found that in order to generate the same output from the xof we needed to first rotate the permutation once by supplying an empty vector before actually using the function.
We also found that the merkle tree was not using the salt.

\section{Benchmarking}\label{sub:benchmarking} % (fold)
We utilized Criterion.rs~\cite{criterion} for initial benchmarking. Criterion.rs is a statistics-driven benchmarking library for Rust, offering robust tools to measure and analyze performance reliably.

\subsection{Benchmark setup}
Benchmarking functions are organized in a separate folder named bench. Within this folder, we create files dedicated to specific benchmarking tasks. To enable benchmarking, the target functions must be declared as pub (public) and exported through a lib.rs file. This structure creates a package that can be accessed and invoked by the benchmarking files.

\subsection{Test Environment}
Benchmarks were conducted on two machines: a Dell XPS 15 9510 and a MacBook Pro with an M2 Pro chip. These machines represent different architectures, allowing us to observe subtle variations in runtime performance based on hardware differences. The specifications of the machines are as follows:

\begin{itemize}
  \item \textbf{Dell XPS 15 9510}:
        \begin{itemize}
          \item \textbf{CPU}: 11th Gen Intel Core i7-11800H @ 2.30GHz x 16
          \item \textbf{GPU}: GeForce RTX 3050 Ti Mobile
          \item \textbf{RAM}: 32GB DDR4 3200MHz
          \item \textbf{OS}: Ubuntu 22.04.5 LTS, 64-bit
        \end{itemize}
  \item \textbf{MacBook Pro (M2 Pro)}:
        \begin{itemize}
          \item \textbf{CPU}: Apple M2 Pro (10-core CPU: 6 performance cores, 4 efficiency cores)
          \item \textbf{RAM}: 32GB LPDDR5
          \item \textbf{OS}: macOS Sequoia 15.1.1 (24B91)
        \end{itemize}
\end{itemize}
This setup provided a diverse testing environment, ensuring our benchmarks accounted for both x86\_64 and ARM architectures.
% subsection Benchmarking (end)

\subsection{Feature flags}\label{sub:feature_flags} % (fold)
To maintain a baseline implementation while enabling flexibility to switch between different NIST categories, we utilized Rust's feature flags. These flags support conditional compilation, allowing multiple implementations of the same function to coexist in the codebase. Only the specific implementation required is compiled, based on the flag provided to the compiler. This approach made it possible to test individual optimizations in isolation, rather than applying all optimizations simultaneously.

The way we implemented this was by using the \texttt{cfg} attribute. This attribute allows us to specify a feature flag, which can be used to conditionally compile code. For example, we can use the following code to conditionally compile a function:
\begin{minted}{rust}
#[cfg(not(feature = "simd"))]
pub fn gf256_add_vector(vz: &mut [u8], vx: &[u8]) {
... add vector the naive way ...
}

#[cfg(feature = "simd")]
pub fn gf256_add_vector(vz: &mut [u8], vx: &[u8]) {
... add vector the simd way ...
}
\end{minted}
This ensures that when the feature flag for $simd$ is set the compiled code will have the simd function only.

\section{Categories}\label{sub:categories} % (fold)
We tried reworking to use dynamically sized vectors instead of compile time sized arrays. Which proved to be a big rework and refactor. Benching showed slower running times. So instead we went with compile time script to set the constants.

This allows us to change between categories in the following ways:
\begin{itemize}
  \item By environment variable
  \item By Feature flag
\end{itemize}

\subsubsection{Environment variables}\label{sub:env_vars}
Environment variables can be set in the command line or given to the compiler when invoking cargo. The following code snippet shows how you would pass the category to the compiler. Due to the fact that we went with
\begin{minted}{bash}
~$ SDITH_CATEGORY=5 cargo bench
\end{minted}

\subsubsection{Feature flags categories}\label{sub:feature_flags_categories}
After implementing the environment variables, we realised that we could also use feature flags to change the category. This is done by adding the following to the Cargo.toml file.
\begin{minted}{rust}
[features]
default = ["category_one", "optimized"]

# Categories
category_one = []
category_three = []
\end{minted}
Where we state the default category to be one, and the optimized flag to be enabled.
Now in the build.rs file we can change the category based on the feature flag in the following way:
\begin{minted}{rust}
fn get_feature_flag_category() -> Result<Category, String> {
    if cfg!(feature = "category_one") {
        Ok(CATEGORY_ONE)
    } else if cfg!(feature = "category_three") {
        Ok(CATEGORY_THREE)
    } else if cfg!(feature = "category_five") {
        Ok(CATEGORY_FIVE)
    } else {
        Err("No category feature flag set".to_string())
    }
}
\end{minted}
One can now invoke the benchmarking for category one by running the following command:
\begin{minted}{bash}
  ~$ cargo bench --features category_one
\end{minted}

\section{Optimizations}
In order to find optimizations we used the samply CPU profiler~\cite{Stange2024mstange}, to locate long running functions and loops. This was done by building the binary with cargo and supplying it to the samply CPU profiler. The profiler then generates seeds $sd_{e,i}$ for each $i \in [N]$ using $\Pi_C$. a call stack and a flamegraph from the binary, this shows how many cycles of the cpu were used in different sections of the code.
\todo{Maybe a picture of how the base program looked in the call stack}


\subsection{Parallelisation}\label{sub:rayon} % (fold)

% subsection Rayon (end)

\subsection{SIMD}\label{sub:simd} % (fold)

% subsection SIMD (end)

\subsection{Hash functions}
Blake3, Haraka v2, KangarooTwelve, Xoodyak

\section{Kat for NIST}\label{sub:kat_for_nist} % (fold)

% subsection Kat for NIST (end)

%%%%%%%%%%%%%%%%%%%%%%%%%%%%%%%%%%%%%%%%%%%%%%%%%%%%%%%%%%%%%%%%%%%%%%%

\chapter{Benchmarks}\label{ch:bench}
diaries of benchmarks.
discussion of results

\todo{old benchmarks \url{https://asecuritysite.com/openssl/openssl3_b2}}

Test on both mac and linux.
nightly vs stable rust
different hashes
benching at different tags
parallelisation, test for amount of cores, 2, 4, 8, 16
no turbo boost (max 2.6 GHz)
cycles per bytes
compare ours to the optimised reference

%%%%%%%%%%%%%%%%%%%%%%%%%%%%%%%%%%%%%%%%%%%%%%%%%%%%%%%%%%%%%%%%%%%%%%%

\chapter{Conclusion}\label{ch:conclusion}

wrap up and pose future work
what should people continue with
point to round 2 NIST
work in context of timeline

\todo{conclude on the problem statement from the introduction}

future work: Verkle trees optimisation

%%%%%%%%%%%%%%%%%%%%%%%%%%%%%%%%%%%%%%%%%%%%%%%%%%%%%%%%%%%%%%%%%%%%%%%

\cleardoublepage
\addcontentsline{toc}{chapter}{Bibliography}
\bibliographystyle{plain}
\bibliography{refs}

%%%%%%%%%%%%%%%%%%%%%%%%%%%%%%%%%%%%%%%%%%%%%%%%%%%%%%%%%%%%%%%%%%%%%%%

\cleardoublepage
\appendix
\chapter{The Technical Details}

\todo{\dots}

\end{document}
